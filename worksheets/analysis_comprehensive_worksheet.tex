% Yolymatics Tutorials — Analysis Comprehensive Worksheet
% Build with: pdflatex analysis_comprehensive_worksheet.tex

\documentclass[11pt,letterpaper]{article}

% Core packages
\usepackage[margin=1in]{geometry}
\usepackage[T1]{fontenc}
\usepackage[utf8]{inputenc}
\usepackage{lmodern}
\usepackage{amsmath, amssymb, amsthm}
\usepackage{graphicx}
\usepackage{enumitem}
\usepackage{fancyhdr}
\usepackage{tcolorbox}
\usepackage{tikz}
\usepackage{hyperref}

% Professional brand colors
\definecolor{YolyPrimary}{HTML}{1E3A8A}   % deep professional blue
\definecolor{YolyAccent}{HTML}{F59E0B}    % warm amber
\definecolor{YolyDark}{HTML}{0F172A}      % slate dark
\definecolor{YolyLight}{HTML}{F8FAFC}     % very light gray

% Header and footer
\pagestyle{fancy}
\fancyhf{}
\fancyhead[L]{\textcolor{YolyPrimary}{\textbf{Yolymatics Tutorials}}}
\fancyhead[R]{\textcolor{YolyDark}{Analysis Comprehensive Worksheet}}
\fancyfoot[C]{\textcolor{YolyDark}{\thepage}}
\fancyfoot[R]{\textcolor{YolyAccent}{\href{mailto:yolymatics007@gmail.com}{yolymatics007@gmail.com}}}
\renewcommand{\headrulewidth}{2pt}
\renewcommand{\footrulewidth}{1pt}
\renewcommand{\headrule}{\hbox to\headwidth{\color{YolyAccent}\leaders\hrule height \headrulewidth\hfill}}
\renewcommand{\footrule}{\hbox to\headwidth{\color{YolyPrimary}\leaders\hrule height \footrulewidth\hfill}}

% Custom tcolorbox styles
\tcbuselibrary{skins,breakable}

\newtcolorbox{sectionbox}[1]{
  colback=YolyPrimary!5,
  colframe=YolyPrimary,
  fonttitle=\bfseries\large,
  title=#1,
  boxrule=2pt,
  arc=3mm,
  breakable
}

\newtcolorbox{problembox}[1]{
  colback=white,
  colframe=YolyAccent,
  fonttitle=\bfseries,
  title=#1,
  boxrule=1.5pt,
  arc=2mm,
  breakable
}

% Workspace command
\newcommand{\workspace}[1]{
  \vspace{#1}
  \begin{center}
    \textcolor{YolyLight}{\rule{0.9\textwidth}{0.5pt}}
  \end{center}
}

% Custom commands
\newcommand{\R}{\mathbb{R}}
\newcommand{\N}{\mathbb{N}}
\newcommand{\Z}{\mathbb{Z}}
\newcommand{\dx}{\,dx}
\newcommand{\dy}{\,dy}
\newcommand{\dt}{\,dt}

\title{
  \textcolor{YolyPrimary}{\Huge\textbf{Analysis}}\\
  \textcolor{YolyAccent}{\Large Comprehensive Worksheet}\\[0.5cm]
  \textcolor{YolyDark}{Improper Integrals, Series, and Differential Equations}
}
\author{\textcolor{YolyPrimary}{\textbf{Yolymatics Tutorials}}}
\date{\textcolor{YolyDark}{\today}}

\begin{document}

\maketitle
\thispagestyle{fancy}

\vspace{1cm}
\begin{center}
  \begin{tcolorbox}[colback=YolyLight,colframe=YolyPrimary,width=0.8\textwidth,arc=3mm]
    \centering
    \textcolor{YolyDark}{\textbf{Student Name:} \underline{\hspace{6cm}}}\\[0.3cm]
    \textcolor{YolyDark}{\textbf{Date:} \underline{\hspace{6cm}}}
  \end{tcolorbox}
\end{center}

\vspace{0.5cm}
\begin{center}
  \textcolor{YolyDark}{\textit{Show all work clearly. Use additional paper if needed.}}
\end{center}

\newpage

% ===========================
% SECTION 1: IMPROPER INTEGRALS
% ===========================
\begin{sectionbox}{Section 7.8 | Improper Integrals}
\vspace{0.3cm}

\begin{problembox}{Problem 1: Type I — Infinite interval}
Determine whether the following improper integrals converge or diverge. If convergent, evaluate.

\textbf{(a)} $\displaystyle \int_1^\infty \frac{1}{x^2} \dx$

\workspace{4cm}

\textbf{(b)} $\displaystyle \int_0^\infty e^{-3x} \dx$

\workspace{4cm}

\textbf{(c)} $\displaystyle \int_{-\infty}^0 \frac{1}{1+x^2} \dx$

\workspace{4cm}
\end{problembox}

\newpage

\begin{problembox}{Problem 2: Type II — Discontinuous integrand}
Evaluate the following improper integrals or show that they diverge.

\textbf{(a)} $\displaystyle \int_0^1 \frac{1}{\sqrt{x}} \dx$

\workspace{5cm}

\textbf{(b)} $\displaystyle \int_0^3 \frac{1}{(x-1)^{2/3}} \dx$

\workspace{5cm}

\textbf{(c)} $\displaystyle \int_{-1}^1 \frac{1}{x^2} \dx$

\workspace{5cm}
\end{problembox}

\newpage

\begin{problembox}{Problem 3: Comparison test for improper integrals}
Use the comparison test to determine whether the integral converges or diverges.

\textbf{(a)} $\displaystyle \int_1^\infty \frac{1+\sin^2 x}{x^2} \dx$

\workspace{6cm}

\textbf{(b)} $\displaystyle \int_2^\infty \frac{1}{\sqrt{x^3-1}} \dx$

\workspace{6cm}
\end{problembox}

\begin{problembox}{Problem 4: Challenge problem}
Find the value of $p$ for which the integral $\displaystyle \int_e^\infty \frac{1}{x(\ln x)^p} \dx$ converges.

\workspace{6cm}
\end{problembox}

\end{sectionbox}

\newpage

% ===========================
% SECTION 2: BETA AND GAMMA FUNCTIONS
% ===========================
\begin{sectionbox}{Beta and Gamma Functions}
\vspace{0.3cm}

\textbf{Recall:}
\begin{itemize}
  \item Gamma function: $\Gamma(n) = \int_0^\infty x^{n-1}e^{-x} \dx$, with $\Gamma(n+1) = n\Gamma(n)$ and $\Gamma(n+1) = n!$ for $n \in \N$
  \item Beta function: $B(m,n) = \int_0^1 x^{m-1}(1-x)^{n-1} \dx = \frac{\Gamma(m)\Gamma(n)}{\Gamma(m+n)}$
\end{itemize}

\vspace{0.5cm}

\begin{problembox}{Problem 5: Gamma function properties}
\textbf{(a)} Show that $\Gamma(1) = 1$ and find $\Gamma(5)$.

\workspace{5cm}

\textbf{(b)} Prove that $\Gamma\left(\frac{1}{2}\right) = \sqrt{\pi}$ by evaluating $\displaystyle \int_0^\infty x^{-1/2}e^{-x} \dx$.

\workspace{6cm}

\textbf{(c)} Evaluate $\Gamma\left(\frac{7}{2}\right)$.

\workspace{4cm}
\end{problembox}

\newpage

\begin{problembox}{Problem 6: Beta function evaluation}
\textbf{(a)} Evaluate $B(3,4)$ using the definition.

\workspace{5cm}

\textbf{(b)} Verify your answer using the relationship $B(m,n) = \frac{\Gamma(m)\Gamma(n)}{\Gamma(m+n)}$.

\workspace{5cm}

\textbf{(c)} Use the beta function to evaluate $\displaystyle \int_0^1 x^5(1-x)^3 \dx$.

\workspace{5cm}
\end{problembox}

\begin{problembox}{Problem 7: Applications}
Express the following integrals in terms of gamma or beta functions, then evaluate:

\textbf{(a)} $\displaystyle \int_0^\infty x^4 e^{-x} \dx$

\workspace{4cm}

\textbf{(b)} $\displaystyle \int_0^{\pi/2} \sin^3\theta \cos^5\theta \, d\theta$ \quad (Hint: Use substitution $x = \sin^2\theta$)

\workspace{6cm}
\end{problembox}

\end{sectionbox}

\newpage

% ===========================
% SECTION 3: SEQUENCES
% ===========================
\begin{sectionbox}{Section 11.1 | Sequences}
\vspace{0.3cm}

\begin{problembox}{Problem 8: Finding limits of sequences}
Find the limit of each sequence or show that it diverges.

\textbf{(a)} $a_n = \frac{3n^2 + 5n}{2n^2 - 1}$

\workspace{4cm}

\textbf{(b)} $a_n = \frac{n^2}{e^n}$

\workspace{5cm}

\textbf{(c)} $a_n = \frac{\ln n}{n}$

\workspace{4cm}

\textbf{(d)} $a_n = \left(1 + \frac{2}{n}\right)^n$

\workspace{5cm}
\end{problembox}

\newpage

\begin{problembox}{Problem 9: Monotonicity and boundedness}
For the sequence $a_n = \frac{n}{n+1}$:

\textbf{(a)} Prove that the sequence is increasing.

\workspace{5cm}

\textbf{(b)} Prove that the sequence is bounded above by 1.

\workspace{4cm}

\textbf{(c)} Find $\lim_{n\to\infty} a_n$.

\workspace{3cm}
\end{problembox}

\begin{problembox}{Problem 10: Recursive sequences}
Consider the sequence defined by $a_1 = 1$ and $a_{n+1} = \sqrt{2 + a_n}$ for $n \geq 1$.

\textbf{(a)} Calculate $a_2, a_3, a_4, a_5$.

\workspace{4cm}

\textbf{(b)} Prove that the sequence is increasing and bounded above by 2.

\workspace{6cm}

\textbf{(c)} Find the limit of the sequence.

\workspace{5cm}
\end{problembox}

\end{sectionbox}

\newpage

% ===========================
% SECTION 4: SERIES
% ===========================
\begin{sectionbox}{Section 11.2 | Series}
\vspace{0.3cm}

\begin{problembox}{Problem 11: Geometric series}
Determine whether each geometric series converges or diverges. If it converges, find the sum.

\textbf{(a)} $\displaystyle \sum_{n=1}^\infty \left(\frac{2}{3}\right)^n$

\workspace{4cm}

\textbf{(b)} $\displaystyle \sum_{n=0}^\infty \frac{3^n}{4^{n-1}}$

\workspace{5cm}

\textbf{(c)} $\displaystyle \sum_{n=1}^\infty (-1)^n \left(\frac{5}{4}\right)^n$

\workspace{4cm}
\end{problembox}

\newpage

\begin{problembox}{Problem 12: Telescoping series}
Find the sum of the series by writing out several terms and identifying the pattern.

\textbf{(a)} $\displaystyle \sum_{n=1}^\infty \frac{1}{n(n+1)}$

\workspace{6cm}

\textbf{(b)} $\displaystyle \sum_{n=1}^\infty \left(\frac{1}{\sqrt{n}} - \frac{1}{\sqrt{n+1}}\right)$

\workspace{6cm}
\end{problembox}

\begin{problembox}{Problem 13: Divergence test}
Use the divergence test to show that the following series diverge.

\textbf{(a)} $\displaystyle \sum_{n=1}^\infty \frac{n^2}{n^2+1}$

\workspace{4cm}

\textbf{(b)} $\displaystyle \sum_{n=1}^\infty \frac{n+1}{2n+1}$

\workspace{4cm}
\end{problembox}

\end{sectionbox}

\newpage

% ===========================
% SECTION 5: INTEGRAL TEST
% ===========================
\begin{sectionbox}{Section 11.3 | Integral Test}
\vspace{0.3cm}

\begin{problembox}{Problem 14: Using the integral test}
Use the integral test to determine whether the series converges or diverges.

\textbf{(a)} $\displaystyle \sum_{n=1}^\infty \frac{1}{n^3}$

\workspace{6cm}

\textbf{(b)} $\displaystyle \sum_{n=1}^\infty \frac{n}{n^2+1}$

\workspace{6cm}

\textbf{(c)} $\displaystyle \sum_{n=2}^\infty \frac{1}{n\ln n}$

\workspace{7cm}
\end{problembox}

\newpage

\begin{problembox}{Problem 15: The $p$-series}
\textbf{(a)} State the $p$-series test and explain when a $p$-series converges.

\workspace{3cm}

\textbf{(b)} Determine convergence or divergence: $\displaystyle \sum_{n=1}^\infty \frac{1}{n^{3/2}}$

\workspace{3cm}

\textbf{(c)} For what values of $p$ does $\displaystyle \sum_{n=1}^\infty \frac{1}{n^p}$ converge?

\workspace{3cm}
\end{problembox}

\begin{problembox}{Problem 16: Estimating sums}
Consider the series $\displaystyle \sum_{n=1}^\infty \frac{1}{n^4}$.

\textbf{(a)} Use the integral test remainder estimate to find bounds for the error if we approximate the sum by $S_{10}$ (sum of first 10 terms).

\workspace{6cm}

\textbf{(b)} How many terms are needed to ensure the error is less than 0.0001?

\workspace{5cm}
\end{problembox}

\end{sectionbox}

\newpage

% ===========================
% SECTION 6: COMPARISON TEST
% ===========================
\begin{sectionbox}{Section 11.4 | Comparison Test}
\vspace{0.3cm}

\begin{problembox}{Problem 17: Direct comparison test}
Use the direct comparison test to determine whether the series converges or diverges.

\textbf{(a)} $\displaystyle \sum_{n=1}^\infty \frac{1}{2^n + n}$

\workspace{6cm}

\textbf{(b)} $\displaystyle \sum_{n=1}^\infty \frac{n+1}{n^3 - n}$

\workspace{6cm}

\textbf{(c)} $\displaystyle \sum_{n=1}^\infty \frac{\sin^2 n}{n^2}$

\workspace{6cm}
\end{problembox}

\newpage

\begin{problembox}{Problem 18: Limit comparison test}
Use the limit comparison test to determine convergence or divergence.

\textbf{(a)} $\displaystyle \sum_{n=1}^\infty \frac{2n^2 + 3n}{5n^4 + n^2 - 1}$

\workspace{7cm}

\textbf{(b)} $\displaystyle \sum_{n=1}^\infty \frac{\sqrt{n+1}}{n^2}$

\workspace{7cm}

\textbf{(c)} $\displaystyle \sum_{n=1}^\infty \frac{1}{\sqrt{n^3 + 2n}}$

\workspace{7cm}
\end{problembox}

\end{sectionbox}

\newpage

% ===========================
% SECTION 7: ALTERNATING SERIES
% ===========================
\begin{sectionbox}{Section 11.5 | Alternating Series Test and Absolute Convergence}
\vspace{0.3cm}

\begin{problembox}{Problem 19: Alternating series test}
Determine whether the alternating series converges or diverges.

\textbf{(a)} $\displaystyle \sum_{n=1}^\infty (-1)^{n+1} \frac{1}{n}$ (alternating harmonic series)

\workspace{5cm}

\textbf{(b)} $\displaystyle \sum_{n=1}^\infty (-1)^n \frac{n}{n^2+1}$

\workspace{6cm}

\textbf{(c)} $\displaystyle \sum_{n=1}^\infty (-1)^{n+1} \frac{n^2}{n^3+1}$

\workspace{6cm}
\end{problembox}

\newpage

\begin{problembox}{Problem 20: Absolute vs conditional convergence}
Determine whether the series is absolutely convergent, conditionally convergent, or divergent.

\textbf{(a)} $\displaystyle \sum_{n=1}^\infty (-1)^n \frac{1}{n^2}$

\workspace{6cm}

\textbf{(b)} $\displaystyle \sum_{n=1}^\infty \frac{(-1)^{n+1}}{n}$

\workspace{6cm}

\textbf{(c)} $\displaystyle \sum_{n=1}^\infty \frac{(-1)^n n}{n+1}$

\workspace{6cm}
\end{problembox}

\begin{problembox}{Problem 21: Error estimation}
For the series $\displaystyle \sum_{n=1}^\infty \frac{(-1)^{n+1}}{n^3}$:

\textbf{(a)} Estimate the sum using the first 5 terms.

\workspace{4cm}

\textbf{(b)} Use the alternating series estimation theorem to find a bound for the error.

\workspace{4cm}
\end{problembox}

\end{sectionbox}

\newpage

% ===========================
% SECTION 8: RATIO AND ROOT TEST
% ===========================
\begin{sectionbox}{Section 11.6 | Ratio and Root Test}
\vspace{0.3cm}

\begin{problembox}{Problem 22: Ratio test}
Use the ratio test to determine whether the series converges or diverges.

\textbf{(a)} $\displaystyle \sum_{n=1}^\infty \frac{n!}{n^n}$

\workspace{7cm}

\textbf{(b)} $\displaystyle \sum_{n=1}^\infty \frac{2^n}{n!}$

\workspace{7cm}

\textbf{(c)} $\displaystyle \sum_{n=1}^\infty \frac{(n!)^2}{(2n)!}$

\workspace{7cm}
\end{problembox}

\newpage

\begin{problembox}{Problem 23: Root test}
Use the root test to determine convergence or divergence.

\textbf{(a)} $\displaystyle \sum_{n=1}^\infty \left(\frac{2n+3}{3n+2}\right)^n$

\workspace{6cm}

\textbf{(b)} $\displaystyle \sum_{n=1}^\infty \frac{n^n}{(n+1)^{2n}}$

\workspace{7cm}
\end{problembox}

\begin{problembox}{Problem 24: Choosing the right test}
For each series, state which test you would use and determine convergence/divergence.

\textbf{(a)} $\displaystyle \sum_{n=1}^\infty \frac{n^2 3^n}{n!}$

\workspace{6cm}

\textbf{(b)} $\displaystyle \sum_{n=1}^\infty \frac{(-1)^n}{\sqrt{n}}$

\workspace{5cm}
\end{problembox}

\end{sectionbox}

\newpage

% ===========================
% SECTION 9: STRATEGY FOR TESTING SERIES
% ===========================
\begin{sectionbox}{Section 11.7 | Strategy for Testing Series (Extra Practice)}
\vspace{0.3cm}

\begin{problembox}{Problem 25: Mixed practice I}
Determine whether each series converges or diverges. State which test you use.

\textbf{(a)} $\displaystyle \sum_{n=1}^\infty \frac{n^2 + 1}{n^4 + n}$

\workspace{6cm}

\textbf{(b)} $\displaystyle \sum_{n=1}^\infty \frac{1 \cdot 3 \cdot 5 \cdots (2n-1)}{n!}$

\workspace{7cm}

\textbf{(c)} $\displaystyle \sum_{n=1}^\infty \frac{\cos n}{n^3}$

\workspace{6cm}
\end{problembox}

\newpage

\begin{problembox}{Problem 26: Mixed practice II}
Determine convergence or divergence. Justify your answer.

\textbf{(a)} $\displaystyle \sum_{n=1}^\infty \frac{n!}{2 \cdot 5 \cdot 8 \cdots (3n-1)}$

\workspace{7cm}

\textbf{(b)} $\displaystyle \sum_{n=1}^\infty \frac{(-1)^n \sqrt{n}}{n+1}$

\workspace{7cm}

\textbf{(c)} $\displaystyle \sum_{n=1}^\infty \frac{n^2 + 2^n}{n^2 + 3^n}$

\workspace{7cm}
\end{problembox}

\newpage

\begin{problembox}{Problem 27: Challenging series}
Determine whether the following series converge or diverge.

\textbf{(a)} $\displaystyle \sum_{n=1}^\infty \frac{2^n n!}{n^n}$

\workspace{8cm}

\textbf{(b)} $\displaystyle \sum_{n=2}^\infty \frac{1}{(\ln n)^n}$

\workspace{8cm}

\textbf{(c)} $\displaystyle \sum_{n=1}^\infty \frac{n! \, e^n}{n^n}$

\workspace{8cm}
\end{problembox}

\end{sectionbox}

\newpage

% ===========================
% SECTION 10: POWER SERIES
% ===========================
\begin{sectionbox}{Section 11.8 | Power Series}
\vspace{0.3cm}

\begin{problembox}{Problem 28: Radius and interval of convergence}
Find the radius of convergence and interval of convergence for each power series.

\textbf{(a)} $\displaystyle \sum_{n=1}^\infty \frac{x^n}{n}$

\workspace{7cm}

\textbf{(b)} $\displaystyle \sum_{n=0}^\infty \frac{(x-2)^n}{n^2 + 1}$

\workspace{8cm}

\textbf{(c)} $\displaystyle \sum_{n=1}^\infty \frac{(-1)^n x^n}{n \cdot 2^n}$

\workspace{8cm}
\end{problembox}

\newpage

\begin{problembox}{Problem 29: Power series with factorials}
Find the radius and interval of convergence.

\textbf{(a)} $\displaystyle \sum_{n=0}^\infty \frac{n! \, x^n}{(2n)!}$

\workspace{8cm}

\textbf{(b)} $\displaystyle \sum_{n=1}^\infty \frac{(x+3)^n}{n \cdot 5^n}$

\workspace{8cm}
\end{problembox}

\begin{problembox}{Problem 30: Testing endpoints}
For the power series $\displaystyle \sum_{n=1}^\infty \frac{x^n}{n^3}$:

\textbf{(a)} Find the radius of convergence.

\workspace{5cm}

\textbf{(b)} Test convergence at both endpoints.

\workspace{7cm}

\textbf{(c)} State the interval of convergence.

\workspace{2cm}
\end{problembox}

\end{sectionbox}

\newpage

% ===========================
% SECTION 11: REPRESENTATIONS OF FUNCTIONS
% ===========================
\begin{sectionbox}{Section 11.9 | Representations of Functions as Power Series}
\vspace{0.3cm}

\begin{problembox}{Problem 31: Power series representation}
Find a power series representation for each function and determine the radius of convergence.

\textbf{(a)} $f(x) = \frac{1}{1+x}$

\workspace{6cm}

\textbf{(b)} $f(x) = \frac{1}{1-x^2}$

\workspace{6cm}

\textbf{(c)} $f(x) = \frac{x}{(1-x)^2}$

\workspace{7cm}
\end{problembox}

\newpage

\begin{problembox}{Problem 32: Integration and differentiation}
\textbf{(a)} Starting with the geometric series, find a power series for $f(x) = \ln(1+x)$.

\workspace{8cm}

\textbf{(b)} Find a power series representation for $f(x) = \arctan x$.

\workspace{8cm}

\textbf{(c)} Use part (b) to find a series for $\pi$ (Hint: evaluate at $x=1$).

\workspace{6cm}
\end{problembox}

\begin{problembox}{Problem 33: Application}
Use a power series to evaluate the integral $\displaystyle \int_0^{0.5} \frac{\sin x}{x} \dx$ accurate to three decimal places.

\workspace{8cm}
\end{problembox}

\end{sectionbox}

\newpage

% ===========================
% SECTION 12: TAYLOR SERIES
% ===========================
\begin{sectionbox}{Section 11.10 | Taylor and Maclaurin Series}
\vspace{0.3cm}

\begin{problembox}{Problem 34: Finding Maclaurin series}
Find the Maclaurin series for each function.

\textbf{(a)} $f(x) = e^x$

\workspace{6cm}

\textbf{(b)} $f(x) = \sin x$

\workspace{7cm}

\textbf{(c)} $f(x) = \cos x$

\workspace{7cm}
\end{problembox}

\newpage

\begin{problembox}{Problem 35: Taylor series at $x = a$}
Find the Taylor series for $f(x) = \ln x$ centered at $a = 1$.

\workspace{10cm}
\end{problembox}

\begin{problembox}{Problem 36: Using known series}
Use known Maclaurin series to find the Maclaurin series for:

\textbf{(a)} $f(x) = x \cos(x^2)$

\workspace{7cm}

\textbf{(b)} $f(x) = e^{-x^2}$

\workspace{7cm}

\textbf{(c)} $f(x) = \frac{1-\cos x}{x^2}$

\workspace{8cm}
\end{problembox}

\newpage

\begin{problembox}{Problem 37: Taylor polynomial approximation}
\textbf{(a)} Find the third-degree Taylor polynomial $T_3(x)$ for $f(x) = \sqrt{x}$ at $a = 4$.

\workspace{8cm}

\textbf{(b)} Use Taylor's inequality to estimate the error in using $T_3(x)$ to approximate $f(5)$.

\workspace{8cm}
\end{problembox}

\begin{problembox}{Problem 38: Applications of Taylor series}
\textbf{(a)} Use the Maclaurin series for $e^x$ to evaluate $\lim_{x \to 0} \frac{e^x - 1 - x}{x^2}$.

\workspace{6cm}

\textbf{(b)} Approximate $\sqrt{e}$ using the first four terms of the Maclaurin series for $e^{x/2}$.

\workspace{6cm}
\end{problembox}

\end{sectionbox}

\newpage

% ===========================
% SECTION 13: SECOND-ORDER LINEAR DEs
% ===========================
\begin{sectionbox}{Second-Order Linear Differential Equations}
\vspace{0.3cm}

\textbf{Recall:} A second-order linear differential equation has the form:
\[
a(x)y'' + b(x)y' + c(x)y = f(x)
\]
If $f(x) = 0$, the equation is \textbf{homogeneous}. Otherwise, it is \textbf{nonhomogeneous}.

\vspace{0.5cm}

\begin{problembox}{Problem 39: Homogeneous equations with constant coefficients}
Solve the following differential equations.

\textbf{(a)} $y'' - 5y' + 6y = 0$

\workspace{7cm}

\textbf{(b)} $y'' + 4y' + 4y = 0$

\workspace{7cm}

\textbf{(c)} $y'' + 2y' + 5y = 0$

\workspace{8cm}
\end{problembox}

\newpage

\begin{problembox}{Problem 40: Initial value problems}
Solve the initial value problem.

\textbf{(a)} $y'' - 3y' - 4y = 0$, \quad $y(0) = 2$, \quad $y'(0) = -1$

\workspace{10cm}

\textbf{(b)} $y'' + 9y = 0$, \quad $y(0) = 1$, \quad $y'(0) = 3$

\workspace{10cm}
\end{problembox}

\newpage

\begin{problembox}{Problem 41: Characteristic equation analysis}
For each differential equation, find the characteristic equation, classify the roots, and write the general solution.

\textbf{(a)} $y'' - 6y' + 9y = 0$

\workspace{7cm}

\textbf{(b)} $y'' + y' - 6y = 0$

\workspace{7cm}

\textbf{(c)} $y'' - 2y' + 10y = 0$

\workspace{7cm}
\end{problembox}

\end{sectionbox}

\newpage

% ===========================
% SECTION 14: NONHOMOGENEOUS LINEAR EQUATIONS
% ===========================
\begin{sectionbox}{Nonhomogeneous Linear Equations}
\vspace{0.3cm}

\textbf{Method:} The general solution is $y = y_h + y_p$, where:
\begin{itemize}
  \item $y_h$ is the general solution to the homogeneous equation
  \item $y_p$ is a particular solution to the nonhomogeneous equation
\end{itemize}

\vspace{0.5cm}

\begin{problembox}{Problem 42: Method of undetermined coefficients}
Solve the differential equation using the method of undetermined coefficients.

\textbf{(a)} $y'' - 3y' + 2y = e^{3x}$

\workspace{10cm}

\textbf{(b)} $y'' + 4y = 8\sin(2x)$

\workspace{10cm}
\end{problembox}

\newpage

\begin{problembox}{Problem 43: Polynomial and exponential forcing}
Find the general solution.

\textbf{(a)} $y'' - y' - 2y = 4x^2$

\workspace{11cm}

\textbf{(b)} $y'' + y = x e^x$

\workspace{11cm}
\end{problembox}

\newpage

\begin{problembox}{Problem 44: Variation of parameters}
Use the method of variation of parameters to solve:

\textbf{(a)} $y'' + y = \sec x$, \quad $0 < x < \pi/2$

\workspace{12cm}

\textbf{(b)} $y'' - 2y' + y = \frac{e^x}{x}$, \quad $x > 0$

\workspace{12cm}
\end{problembox}

\newpage

\begin{problembox}{Problem 45: Initial value problem — Nonhomogeneous}
Solve the initial value problem.

$y'' + 4y' + 4y = e^{-2x}$, \quad $y(0) = 1$, \quad $y'(0) = 0$

\workspace{14cm}
\end{problembox}

\begin{problembox}{Problem 46: Application — Spring-mass system}
A mass on a spring satisfies the differential equation:
\[
m\frac{d^2x}{dt^2} + c\frac{dx}{dt} + kx = F(t)
\]

For $m = 1$ kg, $c = 4$ kg/s, $k = 5$ N/m, and external force $F(t) = 10\cos(2t)$ N:

\textbf{(a)} Write the differential equation.

\workspace{3cm}

\textbf{(b)} Find the general solution.

\workspace{12cm}

\textbf{(c)} If $x(0) = 0$ and $x'(0) = 0$, find the particular solution.

\workspace{8cm}
\end{problembox}

\end{sectionbox}

\newpage

% ===========================
% SUMMARY PAGE
% ===========================
\begin{center}
  \begin{tcolorbox}[colback=YolyPrimary!10,colframe=YolyPrimary,width=0.9\textwidth,arc=4mm,boxrule=2pt]
    \begin{center}
      {\Large\textcolor{YolyPrimary}{\textbf{Excellent Work!}}}\\[0.5cm]
      \textcolor{YolyDark}{You have completed a comprehensive review of Analysis topics.}\\[0.3cm]
      \textcolor{YolyDark}{Remember to review any problems you found challenging.}\\[1cm]
      {\large\textcolor{YolyAccent}{\textbf{Yolymatics Tutorials}}}\\
      \textcolor{YolyDark}{\href{mailto:yolymatics007@gmail.com}{yolymatics007@gmail.com}}
    \end{center}
  \end{tcolorbox}
\end{center}

\vfill

\begin{center}
  \textcolor{YolyDark}{\textit{For additional practice problems or tutoring, contact us!}}
\end{center}

\end{document}
