\documentclass[aspectratio=169]{beamer}
\usetheme{Madrid}
\usecolortheme{default}

% Packages
\usepackage{amsmath}
\usepackage{amssymb}
\usepackage{tikz}
\usepackage{array}
\usepackage{booktabs}

% Define Yolymatics colors
\definecolor{yolyblue}{RGB}{13,71,161}
\definecolor{yolylightblue}{RGB}{25,118,210}
\definecolor{yolyaccent}{RGB}{245,124,0}

% Set theme colors
\setbeamercolor{structure}{fg=yolyblue}
\setbeamercolor{palette primary}{bg=yolyblue,fg=white}
\setbeamercolor{palette secondary}{bg=yolylightblue,fg=white}
\setbeamercolor{palette tertiary}{bg=yolyaccent,fg=white}
\setbeamercolor{frametitle}{bg=yolyblue,fg=white}

% Footer
\setbeamertemplate{footline}{
  \leavevmode%
  \hbox{%
  \begin{beamercolorbox}[wd=.333333\paperwidth,ht=2.25ex,dp=1ex,center]{author in head/foot}%
    \usebeamerfont{author in head/foot}\insertshortauthor
  \end{beamercolorbox}%
  \begin{beamercolorbox}[wd=.333333\paperwidth,ht=2.25ex,dp=1ex,center]{title in head/foot}%
    \usebeamerfont{title in head/foot}\insertshorttitle
  \end{beamercolorbox}%
  \begin{beamercolorbox}[wd=.333333\paperwidth,ht=2.25ex,dp=1ex,right]{date in head/foot}%
    \usebeamerfont{date in head/foot}\insertshortdate{}\hspace*{2em}
    \insertframenumber{} / \inserttotalframenumber\hspace*{2ex} 
  \end{beamercolorbox}}%
  \vskip0pt%
}

% Title information
\title{Improper Integrals}
\subtitle{Practice Problems}
\author{Yolymatics Tutorials}
\institute{www.yolymaticstutorials.com}
\date{\today}

\begin{document}

% Title slide
\begin{frame}
  \titlepage
\end{frame}

% Table of contents
\begin{frame}{Outline}
  \tableofcontents
\end{frame}

% Section 1: Introduction
\section{Introduction to Improper Integrals}

\begin{frame}{What are Improper Integrals?}
  \begin{block}{Definition}
    An \textbf{improper integral} is a definite integral that has one or both of the following properties:
    \begin{itemize}
      \item One or both limits of integration are infinite
      \item The integrand has an infinite discontinuity in the interval of integration
    \end{itemize}
  \end{block}
  
  \vspace{0.5cm}
  
  \begin{block}{Types of Improper Integrals}
    \begin{enumerate}
      \item \textbf{Type 1:} Infinite limits of integration
      \item \textbf{Type 2:} Infinite discontinuity in the integrand
    \end{enumerate}
  \end{block}
\end{frame}

\begin{frame}{Type 1: Infinite Limits}
  \begin{block}{Upper Limit Infinite}
    $$\int_a^{\infty} f(x)\,dx = \lim_{t \to \infty} \int_a^t f(x)\,dx$$
  \end{block}
  
  \begin{block}{Lower Limit Infinite}
    $$\int_{-\infty}^b f(x)\,dx = \lim_{t \to -\infty} \int_t^b f(x)\,dx$$
  \end{block}
  
  \begin{block}{Both Limits Infinite}
    $$\int_{-\infty}^{\infty} f(x)\,dx = \int_{-\infty}^c f(x)\,dx + \int_c^{\infty} f(x)\,dx$$
    where $c$ is any real number. Both integrals must converge.
  \end{block}
\end{frame}

\begin{frame}{Type 2: Infinite Discontinuity}
  \begin{block}{Discontinuity at Upper Limit}
    If $f$ is continuous on $[a,b)$ and discontinuous at $b$:
    $$\int_a^b f(x)\,dx = \lim_{t \to b^-} \int_a^t f(x)\,dx$$
  \end{block}
  
  \begin{block}{Discontinuity at Lower Limit}
    If $f$ is continuous on $(a,b]$ and discontinuous at $a$:
    $$\int_a^b f(x)\,dx = \lim_{t \to a^+} \int_t^b f(x)\,dx$$
  \end{block}
  
  \begin{block}{Discontinuity in the Interior}
    If $f$ is discontinuous at $c$ where $a < c < b$:
    $$\int_a^b f(x)\,dx = \int_a^c f(x)\,dx + \int_c^b f(x)\,dx$$
  \end{block}
\end{frame}

\begin{frame}{Convergence and Divergence}
  \begin{block}{Convergent Improper Integral}
    An improper integral is \textbf{convergent} if the limit defining it exists and is finite.
  \end{block}
  
  \begin{block}{Divergent Improper Integral}
    An improper integral is \textbf{divergent} if the limit does not exist or is infinite.
  \end{block}
  
  \vspace{0.5cm}
  
  \begin{alertblock}{Important Note}
    When an improper integral involves both infinite limits and discontinuities, or multiple discontinuities, each part must be evaluated separately and ALL parts must converge for the integral to be convergent.
  \end{alertblock}
\end{frame}

% Section 2: Type 1 Practice Problems
\section{Type 1: Infinite Limits}

\begin{frame}{Problem 1: Upper Limit Infinite}
  \begin{block}{Problem}
    Evaluate the improper integral:
    $$\int_1^{\infty} \frac{1}{x^2}\,dx$$
    
    Does it converge or diverge? If it converges, find its value.
  \end{block}
\end{frame}

\begin{frame}{Working Space}
  \vspace{6cm}
\end{frame}

\begin{frame}{Problem 2: Upper Limit Infinite}
  \begin{block}{Problem}
    Evaluate the improper integral:
    $$\int_1^{\infty} \frac{1}{x}\,dx$$
    
    Does it converge or diverge? If it converges, find its value.
  \end{block}
\end{frame}

\begin{frame}{Working Space}
  \vspace{6cm}
\end{frame}

\begin{frame}{Problem 3: Upper Limit Infinite with Exponential}
  \begin{block}{Problem}
    Evaluate the improper integral:
    $$\int_0^{\infty} e^{-3x}\,dx$$
    
    Does it converge or diverge? If it converges, find its value.
  \end{block}
\end{frame}

\begin{frame}{Working Space}
  \vspace{6cm}
\end{frame}

\begin{frame}{Problem 4: Lower Limit Infinite}
  \begin{block}{Problem}
    Evaluate the improper integral:
    $$\int_{-\infty}^0 e^{2x}\,dx$$
    
    Does it converge or diverge? If it converges, find its value.
  \end{block}
\end{frame}

\begin{frame}{Working Space}
  \vspace{6cm}
\end{frame}

\begin{frame}{Problem 5: Both Limits Infinite}
  \begin{block}{Problem}
    Evaluate the improper integral:
    $$\int_{-\infty}^{\infty} \frac{1}{1+x^2}\,dx$$
    
    Does it converge or diverge? If it converges, find its value.
  \end{block}
\end{frame}

\begin{frame}{Working Space}
  \vspace{6cm}
\end{frame}

\begin{frame}{Problem 6: p-integral}
  \begin{block}{Problem}
    For what values of $p$ does the following integral converge?
    $$\int_2^{\infty} \frac{1}{x^p}\,dx$$
    
    When it converges, find its value in terms of $p$.
  \end{block}
\end{frame}

\begin{frame}{Working Space}
  \vspace{6cm}
\end{frame}

\begin{frame}{Problem 7: Trigonometric Function}
  \begin{block}{Problem}
    Evaluate the improper integral:
    $$\int_0^{\infty} \sin(x)\,dx$$
    
    Does it converge or diverge? Explain your reasoning.
  \end{block}
\end{frame}

\begin{frame}{Working Space}
  \vspace{6cm}
\end{frame}

% Section 3: Type 2 Practice Problems
\section{Type 2: Infinite Discontinuities}

\begin{frame}{Problem 8: Discontinuity at Upper Limit}
  \begin{block}{Problem}
    Evaluate the improper integral:
    $$\int_0^1 \frac{1}{\sqrt{1-x}}\,dx$$
    
    Does it converge or diverge? If it converges, find its value.
  \end{block}
\end{frame}

\begin{frame}{Working Space}
  \vspace{6cm}
\end{frame}

\begin{frame}{Problem 9: Discontinuity at Lower Limit}
  \begin{block}{Problem}
    Evaluate the improper integral:
    $$\int_0^1 \frac{1}{\sqrt{x}}\,dx$$
    
    Does it converge or diverge? If it converges, find its value.
  \end{block}
\end{frame}

\begin{frame}{Working Space}
  \vspace{6cm}
\end{frame}

\begin{frame}{Problem 10: Discontinuity at Interior Point}
  \begin{block}{Problem}
    Evaluate the improper integral:
    $$\int_0^3 \frac{1}{x-1}\,dx$$
    
    Does it converge or diverge? If it converges, find its value.
    
    \textit{Hint: The integrand has a discontinuity at $x = 1$.}
  \end{block}
\end{frame}

\begin{frame}{Working Space}
  \vspace{6cm}
\end{frame}

\begin{frame}{Problem 11: Logarithmic Integrand}
  \begin{block}{Problem}
    Evaluate the improper integral:
    $$\int_0^1 \ln(x)\,dx$$
    
    Does it converge or diverge? If it converges, find its value.
  \end{block}
\end{frame}

\begin{frame}{Working Space}
  \vspace{6cm}
\end{frame}

\begin{frame}{Problem 12: Power Function with Discontinuity}
  \begin{block}{Problem}
    Evaluate the improper integral:
    $$\int_0^4 \frac{1}{x^{2/3}}\,dx$$
    
    Does it converge or diverge? If it converges, find its value.
  \end{block}
\end{frame}

\begin{frame}{Working Space}
  \vspace{6cm}
\end{frame}

% Section 4: Mixed and Advanced Problems
\section{Mixed and Advanced Problems}

\begin{frame}{Problem 13: Both Infinite Limit and Discontinuity}
  \begin{block}{Problem}
    Evaluate the improper integral:
    $$\int_0^{\infty} \frac{1}{\sqrt{x}(1+x)}\,dx$$
    
    Does it converge or diverge? If it converges, find its value.
    
    \textit{Hint: Split at $x = 1$ and evaluate two improper integrals.}
  \end{block}
\end{frame}

\begin{frame}{Working Space}
  \vspace{6cm}
\end{frame}

\begin{frame}{Problem 14: Comparison Test Application}
  \begin{block}{Problem}
    Determine whether the following integral converges or diverges:
    $$\int_1^{\infty} \frac{1+\sin^2(x)}{x^2}\,dx$$
    
    \textit{Hint: Use the comparison test with an appropriate function.}
  \end{block}
\end{frame}

\begin{frame}{Working Space}
  \vspace{6cm}
\end{frame}

\begin{frame}{Problem 15: Exponential with Polynomial}
  \begin{block}{Problem}
    Evaluate the improper integral:
    $$\int_0^{\infty} x^2 e^{-x}\,dx$$
    
    Does it converge or diverge? If it converges, find its value.
    
    \textit{Hint: Use integration by parts twice.}
  \end{block}
\end{frame}

\begin{frame}{Working Space}
  \vspace{6cm}
\end{frame}

\begin{frame}{Problem 16: Rational Function}
  \begin{block}{Problem}
    Evaluate the improper integral:
    $$\int_2^{\infty} \frac{1}{x^2-1}\,dx$$
    
    Does it converge or diverge? If it converges, find its value.
    
    \textit{Hint: Use partial fractions first.}
  \end{block}
\end{frame}

\begin{frame}{Working Space}
  \vspace{6cm}
\end{frame}

% Section 5: Useful Formulas and Results
\section{Reference Formulas}

\begin{frame}{Important Results for Improper Integrals}
  \begin{block}{p-integrals (Type 1)}
    $$\int_1^{\infty} \frac{1}{x^p}\,dx \text{ converges if } p > 1 \text{ and diverges if } p \leq 1$$
    
    When $p > 1$: $\displaystyle\int_1^{\infty} \frac{1}{x^p}\,dx = \frac{1}{p-1}$
  \end{block}
  
  \begin{block}{p-integrals (Type 2)}
    $$\int_0^1 \frac{1}{x^p}\,dx \text{ converges if } p < 1 \text{ and diverges if } p \geq 1$$
    
    When $p < 1$: $\displaystyle\int_0^1 \frac{1}{x^p}\,dx = \frac{1}{1-p}$
  \end{block}
\end{frame}

\begin{frame}{Common Convergent Integrals}
  \begin{block}{Useful Results}
    \begin{enumerate}
      \item $\displaystyle\int_0^{\infty} e^{-ax}\,dx = \frac{1}{a}$ for $a > 0$
      
      \item $\displaystyle\int_{-\infty}^{\infty} \frac{1}{1+x^2}\,dx = \pi$
      
      \item $\displaystyle\int_0^{\infty} \frac{\sin(x)}{x}\,dx = \frac{\pi}{2}$
      
      \item $\displaystyle\int_0^{\infty} x^n e^{-x}\,dx = n!$ for $n \in \mathbb{N}$
      
      \item $\displaystyle\int_{-\infty}^{\infty} e^{-x^2}\,dx = \sqrt{\pi}$ (Gaussian integral)
    \end{enumerate}
  \end{block}
\end{frame}

\begin{frame}{Comparison Tests}
  \begin{block}{Direct Comparison Test}
    If $0 \leq f(x) \leq g(x)$ for all $x \geq a$:
    \begin{itemize}
      \item If $\int_a^{\infty} g(x)\,dx$ converges, then $\int_a^{\infty} f(x)\,dx$ converges
      \item If $\int_a^{\infty} f(x)\,dx$ diverges, then $\int_a^{\infty} g(x)\,dx$ diverges
    \end{itemize}
  \end{block}
  
  \begin{block}{Limit Comparison Test}
    If $f(x) \geq 0$ and $g(x) > 0$ for $x \geq a$, and:
    $$\lim_{x \to \infty} \frac{f(x)}{g(x)} = L \text{ where } 0 < L < \infty$$
    
    Then $\int_a^{\infty} f(x)\,dx$ and $\int_a^{\infty} g(x)\,dx$ either both converge or both diverge.
  \end{block}
\end{frame}

% Final slide
\begin{frame}
  \begin{center}
    {\Huge Thank You!}
    
    \vspace{1cm}
    
    {\Large Yolymatics Tutorials}
    
    \vspace{0.5cm}
    
    {\large www.yolymaticstutorials.com}
    
    \vspace{1cm}
    
    \textcolor{yolyaccent}{\Large Keep practicing!}
  \end{center}
\end{frame}

\end{document}
