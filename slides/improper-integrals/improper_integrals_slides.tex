\documentclass[aspectratio=169]{beamer}
\usetheme{Madrid}
\usecolortheme{default}

% Packages
\usepackage{amsmath}
\usepackage{amssymb}
\usepackage{tikz}
\usepackage{array}
\usepackage{booktabs}

% Define Yolymatics colors
\definecolor{yolyblue}{RGB}{13,71,161}
\definecolor{yolylightblue}{RGB}{25,118,210}
\definecolor{yolyaccent}{RGB}{245,124,0}

% Set theme colors
\setbeamercolor{structure}{fg=yolyblue}
\setbeamercolor{palette primary}{bg=yolyblue,fg=white}
\setbeamercolor{palette secondary}{bg=yolylightblue,fg=white}
\setbeamercolor{palette tertiary}{bg=yolyaccent,fg=white}
\setbeamercolor{frametitle}{bg=yolyblue,fg=white}

% Footer
\setbeamertemplate{footline}{
  \leavevmode%
  \hbox{%
  \begin{beamercolorbox}[wd=.333333\paperwidth,ht=2.25ex,dp=1ex,center]{author in head/foot}%
    \usebeamerfont{author in head/foot}\insertshortauthor
  \end{beamercolorbox}%
  \begin{beamercolorbox}[wd=.333333\paperwidth,ht=2.25ex,dp=1ex,center]{title in head/foot}%
    \usebeamerfont{title in head/foot}\insertshorttitle
  \end{beamercolorbox}%
  \begin{beamercolorbox}[wd=.333333\paperwidth,ht=2.25ex,dp=1ex,right]{date in head/foot}%
    \usebeamerfont{date in head/foot}\insertshortdate{}\hspace*{2em}
    \insertframenumber{} / \inserttotalframenumber\hspace*{2ex} 
  \end{beamercolorbox}}%
  \vskip0pt%
}

% Title information
\title{Improper Integrals}
\subtitle{Practice Problems}
\author{Yolymatics Tutorials}
\institute{www.yolymaticstutorials.com}
\date{\today}

\begin{document}

% Title slide
\begin{frame}
  \titlepage
\end{frame}

% Table of contents
\begin{frame}{Outline}
  \tableofcontents
\end{frame}

% Section 1: Introduction
\section{Introduction to Improper Integrals}

\begin{frame}{What are Improper Integrals?}
  \begin{block}{Definition}
    An \textbf{improper integral} is a definite integral that has one or both of the following properties:
    \begin{itemize}
      \item One or both limits of integration are infinite
      \item The integrand has an infinite discontinuity in the interval of integration
    \end{itemize}
  \end{block}
  
  \vspace{0.5cm}
  
  \begin{block}{Types of Improper Integrals}
    \begin{enumerate}
      \item \textbf{Type 1:} Infinite limits of integration
      \item \textbf{Type 2:} Infinite discontinuity in the integrand
    \end{enumerate}
  \end{block}
\end{frame}

\begin{frame}{Type 1: Infinite Limits}
  \begin{block}{Upper Limit Infinite}
    $$\int_a^{\infty} f(x)\,dx = \lim_{t \to \infty} \int_a^t f(x)\,dx$$
  \end{block}
  
  \begin{block}{Lower Limit Infinite}
    $$\int_{-\infty}^b f(x)\,dx = \lim_{t \to -\infty} \int_t^b f(x)\,dx$$
  \end{block}
  
  \begin{block}{Both Limits Infinite}
    $$\int_{-\infty}^{\infty} f(x)\,dx = \int_{-\infty}^c f(x)\,dx + \int_c^{\infty} f(x)\,dx$$
    where $c$ is any real number. Both integrals must converge.
  \end{block}
\end{frame}

\begin{frame}{Type 2: Infinite Discontinuity}
  \begin{block}{Discontinuity at Upper Limit}
    If $f$ is continuous on $[a,b)$ and discontinuous at $b$:
    $$\int_a^b f(x)\,dx = \lim_{t \to b^-} \int_a^t f(x)\,dx$$
  \end{block}
  
  \begin{block}{Discontinuity at Lower Limit}
    If $f$ is continuous on $(a,b]$ and discontinuous at $a$:
    $$\int_a^b f(x)\,dx = \lim_{t \to a^+} \int_t^b f(x)\,dx$$
  \end{block}
  
  \begin{block}{Discontinuity in the Interior}
    If $f$ is discontinuous at $c$ where $a < c < b$:
    $$\int_a^b f(x)\,dx = \int_a^c f(x)\,dx + \int_c^b f(x)\,dx$$
  \end{block}
\end{frame}

\begin{frame}{Convergence and Divergence}
  \begin{block}{Convergent Improper Integral}
    An improper integral is \textbf{convergent} if the limit defining it exists and is finite.
  \end{block}
  
  \begin{block}{Divergent Improper Integral}
    An improper integral is \textbf{divergent} if the limit does not exist or is infinite.
  \end{block}
  
  \vspace{0.5cm}
  
  \begin{alertblock}{Important Note}
    When an improper integral involves both infinite limits and discontinuities, or multiple discontinuities, each part must be evaluated separately and ALL parts must converge for the integral to be convergent.
  \end{alertblock}
\end{frame}

% Section 2: Type 1 Practice Problems
\section{Type 1: Infinite Limits}

\begin{frame}{Problem 1: Upper Limit Infinite}
  \begin{block}{Problem}
    Evaluate the improper integral:
    $$\int_1^{\infty} \frac{1}{x^2}\,dx$$
    
    Does it converge or diverge? If it converges, find its value.
  \end{block}
\end{frame}

\begin{frame}{Working Space}
  \vspace{6cm}
\end{frame}

\begin{frame}{Problem 2: Upper Limit Infinite}
  \begin{block}{Problem}
    Evaluate the improper integral:
    $$\int_1^{\infty} \frac{1}{x}\,dx$$
    
    Does it converge or diverge? If it converges, find its value.
  \end{block}
\end{frame}

\begin{frame}{Working Space}
  \vspace{6cm}
\end{frame}

\begin{frame}{Problem 3: Upper Limit Infinite with Exponential}
  \begin{block}{Problem}
    Evaluate the improper integral:
    $$\int_0^{\infty} e^{-3x}\,dx$$
    
    Does it converge or diverge? If it converges, find its value.
  \end{block}
\end{frame}

\begin{frame}{Working Space}
  \vspace{6cm}
\end{frame}

\begin{frame}{Problem 4: Lower Limit Infinite}
  \begin{block}{Problem}
    Evaluate the improper integral:
    $$\int_{-\infty}^0 e^{2x}\,dx$$
    
    Does it converge or diverge? If it converges, find its value.
  \end{block}
\end{frame}

\begin{frame}{Working Space}
  \vspace{6cm}
\end{frame}

\begin{frame}{Problem 5: Both Limits Infinite}
  \begin{block}{Problem}
    Evaluate the improper integral:
    $$\int_{-\infty}^{\infty} \frac{1}{1+x^2}\,dx$$
    
    Does it converge or diverge? If it converges, find its value.
  \end{block}
\end{frame}

\begin{frame}{Working Space}
  \vspace{6cm}
\end{frame}

\begin{frame}{Problem 6: p-integral}
  \begin{block}{Problem}
    For what values of $p$ does the following integral converge?
    $$\int_2^{\infty} \frac{1}{x^p}\,dx$$
    
    When it converges, find its value in terms of $p$.
  \end{block}
\end{frame}

\begin{frame}{Working Space}
  \vspace{6cm}
\end{frame}

\begin{frame}{Problem 7: Convergence Test - Polynomial vs Exponential}
  \begin{block}{Problem}
    Determine whether the following integral converges or diverges:
    $$\int_1^{\infty} \frac{x^3}{e^x}\,dx$$
    
    \textit{Hint: Consider the behavior of exponential vs polynomial functions.}
  \end{block}
\end{frame}

\begin{frame}{Working Space}
  \vspace{6cm}
\end{frame}

% Section 3: Type 2 Practice Problems
\section{Type 2: Infinite Discontinuities}

\begin{frame}{Problem 8: Discontinuity at Upper Limit}
  \begin{block}{Problem}
    Evaluate the improper integral:
    $$\int_0^1 \frac{1}{\sqrt{1-x}}\,dx$$
    
    Does it converge or diverge? If it converges, find its value.
  \end{block}
\end{frame}

\begin{frame}{Working Space}
  \vspace{6cm}
\end{frame}

\begin{frame}{Problem 9: Discontinuity at Lower Limit}
  \begin{block}{Problem}
    Evaluate the improper integral:
    $$\int_0^1 \frac{1}{\sqrt{x}}\,dx$$
    
    Does it converge or diverge? If it converges, find its value.
  \end{block}
\end{frame}

\begin{frame}{Working Space}
  \vspace{6cm}
\end{frame}

\begin{frame}{Problem 10: Discontinuity at Interior Point}
  \begin{block}{Problem}
    Evaluate the improper integral:
    $$\int_0^3 \frac{1}{x-1}\,dx$$
    
    Does it converge or diverge? If it converges, find its value.
    
    \textit{Hint: The integrand has a discontinuity at $x = 1$.}
  \end{block}
\end{frame}

\begin{frame}{Working Space}
  \vspace{6cm}
\end{frame}

\begin{frame}{Problem 11: Logarithmic Integrand}
  \begin{block}{Problem}
    Evaluate the improper integral:
    $$\int_0^1 \ln(x)\,dx$$
    
    Does it converge or diverge? If it converges, find its value.
  \end{block}
\end{frame}

\begin{frame}{Working Space}
  \vspace{6cm}
\end{frame}

\begin{frame}{Problem 12: Power Function with Discontinuity}
  \begin{block}{Problem}
    Evaluate the improper integral:
    $$\int_0^4 \frac{1}{x^{2/3}}\,dx$$
    
    Does it converge or diverge? If it converges, find its value.
  \end{block}
\end{frame}

\begin{frame}{Working Space}
  \vspace{6cm}
\end{frame}

% Section 4: Mixed and Advanced Problems
\section{Mixed and Advanced Problems}

\begin{frame}{Problem 13: Both Infinite Limit and Discontinuity}
  \begin{block}{Problem}
    Evaluate the improper integral:
    $$\int_0^{\infty} \frac{1}{\sqrt{x}(1+x)}\,dx$$
    
    Does it converge or diverge? If it converges, find its value.
    
    \textit{Hint: Split at $x = 1$ and evaluate two improper integrals.}
  \end{block}
\end{frame}

\begin{frame}{Working Space}
  \vspace{6cm}
\end{frame}

\begin{frame}{Problem 14: Comparison Test Application}
  \begin{block}{Problem}
    Determine whether the following integral converges or diverges:
    $$\int_1^{\infty} \frac{1+\sin^2(x)}{x^2}\,dx$$
    
    \textit{Hint: Use the comparison test with an appropriate function.}
  \end{block}
\end{frame}

\begin{frame}{Working Space}
  \vspace{6cm}
\end{frame}

\begin{frame}{Problem 15: Comparison Test - Rational Function}
  \begin{block}{Problem}
    Determine whether the following integral converges or diverges:
    $$\int_2^{\infty} \frac{3x^2 + 5}{x^4 - 1}\,dx$$
    
    \textit{Hint: Compare with $\frac{1}{x^2}$ for large $x$.}
  \end{block}
\end{frame}

\begin{frame}{Working Space}
  \vspace{6cm}
\end{frame}

\begin{frame}{Problem 16: Limit Comparison Test}
  \begin{block}{Problem}
    Determine whether the following integral converges or diverges:
    $$\int_1^{\infty} \frac{\sqrt{x}}{x^3 + 2x + 1}\,dx$$
    
    \textit{Hint: Use the limit comparison test.}
  \end{block}
\end{frame}

\begin{frame}{Working Space}
  \vspace{6cm}
\end{frame}

\begin{frame}{Problem 17: Comparison with Known Divergent}
  \begin{block}{Problem}
    Determine whether the following integral converges or diverges:
    $$\int_2^{\infty} \frac{1}{\sqrt{x} \ln(x)}\,dx$$
    
    \textit{Hint: Compare with $\frac{1}{x}$.}
  \end{block}
\end{frame}

\begin{frame}{Working Space}
  \vspace{6cm}
\end{frame}

\begin{frame}{Problem 18: p-integral Variation}
  \begin{block}{Problem}
    Determine whether the following integral converges or diverges:
    $$\int_1^{\infty} \frac{1}{x(\ln x)^2}\,dx$$
    
    \textit{Hint: Use substitution $u = \ln(x)$.}
  \end{block}
\end{frame}

\begin{frame}{Working Space}
  \vspace{6cm}
\end{frame}

\begin{frame}{Problem 19: Exponential with Polynomial}
  \begin{block}{Problem}
    Determine whether the following integral converges or diverges:
    $$\int_0^{\infty} \frac{x^{10}}{e^{\sqrt{x}}}\,dx$$
    
    \textit{Hint: Consider the comparison test.}
  \end{block}
\end{frame}

\begin{frame}{Working Space}
  \vspace{6cm}
\end{frame}

\begin{frame}{Problem 20: Trigonometric Bounded Function}
  \begin{block}{Problem}
    Determine whether the following integral converges or diverges:
    $$\int_1^{\infty} \frac{\cos^2(x)}{x^{3/2}}\,dx$$
    
    \textit{Hint: Use the fact that $0 \leq \cos^2(x) \leq 1$.}
  \end{block}
\end{frame}

\begin{frame}{Working Space}
  \vspace{6cm}
\end{frame}

\begin{frame}{Problem 21: Algebraic Function}
  \begin{block}{Problem}
    Determine whether the following integral converges or diverges:
    $$\int_1^{\infty} \frac{x+1}{\sqrt{x^5 + 3x^2}}\,dx$$
    
    \textit{Hint: Find the dominant terms for large $x$.}
  \end{block}
\end{frame}

\begin{frame}{Working Space}
  \vspace{6cm}
\end{frame}

\begin{frame}{Problem 22: Rational Function - Borderline Case}
  \begin{block}{Problem}
    Determine whether the following integral converges or diverges:
    $$\int_1^{\infty} \frac{1}{x\sqrt{x^2-1}}\,dx$$
  \end{block}
\end{frame}

\begin{frame}{Working Space}
  \vspace{6cm}
\end{frame}

\begin{frame}{Problem 23: Type 2 - Convergence Test}
  \begin{block}{Problem}
    Determine whether the following integral converges or diverges:
    $$\int_0^1 \frac{x}{\sqrt{1-x^2}}\,dx$$
  \end{block}
\end{frame}

\begin{frame}{Working Space}
  \vspace{6cm}
\end{frame}

\begin{frame}{Problem 24: Type 2 - p-integral at Zero}
  \begin{block}{Problem}
    For what values of $p$ does the following integral converge?
    $$\int_0^1 \frac{\ln(x)}{x^p}\,dx$$
  \end{block}
\end{frame}

\begin{frame}{Working Space}
  \vspace{6cm}
\end{frame}

% Section 5: Convergence Tests Reference
\section{Convergence and Divergence Tests}

\begin{frame}{Test 1: Direct Comparison Test}
  \begin{block}{Statement}
    Suppose $f$ and $g$ are continuous on $[a,\infty)$ with $0 \leq f(x) \leq g(x)$ for all $x \geq a$.
    
    \begin{enumerate}
      \item If $\displaystyle\int_a^{\infty} g(x)\,dx$ converges, then $\displaystyle\int_a^{\infty} f(x)\,dx$ converges
      
      \item If $\displaystyle\int_a^{\infty} f(x)\,dx$ diverges, then $\displaystyle\int_a^{\infty} g(x)\,dx$ diverges
    \end{enumerate}
  \end{block}
  
  \begin{alertblock}{Key Strategy}
    Compare with known convergent/divergent integrals like $\displaystyle\int_a^{\infty} \frac{1}{x^p}\,dx$
  \end{alertblock}
\end{frame}

\begin{frame}{Test 2: Limit Comparison Test}
  \begin{block}{Statement}
    Suppose $f$ and $g$ are continuous, positive functions on $[a,\infty)$.
    
    If $\displaystyle\lim_{x \to \infty} \frac{f(x)}{g(x)} = L$ where $0 < L < \infty$,
    
    then $\displaystyle\int_a^{\infty} f(x)\,dx$ and $\displaystyle\int_a^{\infty} g(x)\,dx$ either both converge or both diverge.
  \end{block}
  
  \begin{alertblock}{When to Use}
    Use when direct comparison is difficult to establish but functions have similar growth rates.
  \end{alertblock}
\end{frame}

\begin{frame}{Test 3: p-integral Test (Type 1)}
  \begin{block}{Statement}
    The integral $\displaystyle\int_1^{\infty} \frac{1}{x^p}\,dx$:
    
    \begin{itemize}
      \item \textbf{Converges} if $p > 1$
      \item \textbf{Diverges} if $p \leq 1$
    \end{itemize}
    
    When $p > 1$: $\displaystyle\int_1^{\infty} \frac{1}{x^p}\,dx = \frac{1}{p-1}$
  \end{block}
  
  \begin{exampleblock}{Extended Form}
    $\displaystyle\int_a^{\infty} \frac{1}{(x-c)^p}\,dx$ has the same convergence properties for $a > c$.
  \end{exampleblock}
\end{frame}

\begin{frame}{Test 4: p-integral Test (Type 2)}
  \begin{block}{Statement}
    The integral $\displaystyle\int_0^1 \frac{1}{x^p}\,dx$:
    
    \begin{itemize}
      \item \textbf{Converges} if $p < 1$
      \item \textbf{Diverges} if $p \geq 1$
    \end{itemize}
    
    When $p < 1$: $\displaystyle\int_0^1 \frac{1}{x^p}\,dx = \frac{1}{1-p}$
  \end{block}
  
  \begin{exampleblock}{Extended Form}
    $\displaystyle\int_a^b \frac{1}{(x-a)^p}\,dx$ converges if $p < 1$, diverges if $p \geq 1$.
    
    $\displaystyle\int_a^b \frac{1}{(b-x)^p}\,dx$ converges if $p < 1$, diverges if $p \geq 1$.
  \end{exampleblock}
\end{frame}

\begin{frame}{Test 5: Logarithmic Integrals}
  \begin{block}{Important Results}
    \begin{enumerate}
      \item $\displaystyle\int_2^{\infty} \frac{1}{x \ln(x)}\,dx$ \textbf{diverges}
      
      \item $\displaystyle\int_2^{\infty} \frac{1}{x (\ln x)^p}\,dx$ \textbf{converges} if $p > 1$, \textbf{diverges} if $p \leq 1$
      
      \item $\displaystyle\int_0^1 \ln(x)\,dx$ \textbf{converges} (equals $-1$)
      
      \item $\displaystyle\int_0^1 |\ln(x)|^p\,dx$ \textbf{converges} for all $p > 0$
    \end{enumerate}
  \end{block}
\end{frame}

\begin{frame}{Test 6: Exponential vs Polynomial}
  \begin{block}{Fundamental Principle}
    For any polynomial $P(x)$ and any constant $a > 0$:
    
    \begin{itemize}
      \item $\displaystyle\int_1^{\infty} P(x)e^{-ax}\,dx$ \textbf{converges}
      
      \item $\displaystyle\int_1^{\infty} \frac{e^{ax}}{P(x)}\,dx$ \textbf{diverges} if $\deg(P) \geq 0$
      
      \item $\displaystyle\lim_{x \to \infty} \frac{P(x)}{e^{ax}} = 0$ (exponential dominates polynomial)
    \end{itemize}
  \end{block}
  
  \begin{alertblock}{Key Insight}
    Exponential decay ($e^{-ax}$) is stronger than any polynomial growth, so it forces convergence.
  \end{alertblock}
\end{frame}

\begin{frame}{Test 7: Bounded Functions}
  \begin{block}{Comparison Strategy}
    If $|f(x)| \leq M$ (bounded) for all $x \geq a$, and $\displaystyle\int_a^{\infty} g(x)\,dx$ converges with $g(x) > 0$:
    
    Then $\displaystyle\int_a^{\infty} f(x)g(x)\,dx$ converges.
  \end{block}
  
  \begin{exampleblock}{Common Applications}
    \begin{itemize}
      \item $|\sin(x)| \leq 1$, $|\cos(x)| \leq 1$
      \item If $\displaystyle\int_a^{\infty} \frac{1}{x^p}\,dx$ converges (i.e., $p > 1$), then $\displaystyle\int_a^{\infty} \frac{\sin(x)}{x^p}\,dx$ converges
    \end{itemize}
  \end{exampleblock}
\end{frame}

\begin{frame}{Test 8: Asymptotic Behavior}
  \begin{block}{Dominant Term Analysis}
    For rational functions, identify the dominant terms as $x \to \infty$:
    
    $$\int_a^{\infty} \frac{a_nx^n + \cdots + a_0}{b_mx^m + \cdots + b_0}\,dx \text{ behaves like } \int_a^{\infty} \frac{a_n}{b_m} x^{n-m}\,dx$$
    
    This converges if $n - m < -1$ (i.e., $m > n + 1$).
  \end{block}
  
  \begin{exampleblock}{Example}
    $\displaystyle\int_1^{\infty} \frac{3x^2 + 5x + 1}{x^4 + 2x - 7}\,dx$ behaves like $\displaystyle\int_1^{\infty} \frac{3}{x^2}\,dx$ which converges.
  \end{exampleblock}
\end{frame}

\begin{frame}{Test 9: Absolute Convergence}
  \begin{block}{Statement}
    If $\displaystyle\int_a^{\infty} |f(x)|\,dx$ converges, then $\displaystyle\int_a^{\infty} f(x)\,dx$ converges.
    
    We say the integral is \textbf{absolutely convergent}.
  \end{block}
  
  \begin{alertblock}{Important Note}
    The converse is NOT true. An integral can converge without converging absolutely (conditional convergence).
    
    Example: $\displaystyle\int_1^{\infty} \frac{\sin(x)}{x}\,dx$ converges, but $\displaystyle\int_1^{\infty} \left|\frac{\sin(x)}{x}\right|\,dx$ diverges.
  \end{alertblock}
\end{frame}

\begin{frame}{Quick Reference: Common Convergent Integrals}
  \begin{block}{Useful Results to Remember}
    \begin{enumerate}
      \item $\displaystyle\int_0^{\infty} e^{-ax}\,dx = \frac{1}{a}$ for $a > 0$
      
      \item $\displaystyle\int_{-\infty}^{\infty} \frac{1}{1+x^2}\,dx = \pi$
      
      \item $\displaystyle\int_0^{\infty} x^n e^{-x}\,dx = n!$ for $n \in \mathbb{N}$
      
      \item $\displaystyle\int_{-\infty}^{\infty} e^{-x^2}\,dx = \sqrt{\pi}$ (Gaussian)
      
      \item $\displaystyle\int_0^{\infty} \frac{\sin(x)}{x}\,dx = \frac{\pi}{2}$
    \end{enumerate}
  \end{block}
\end{frame}

% Section 6: Useful Formulas
\section{Reference Formulas}

\begin{frame}{Important Integration Formulas}
  \begin{block}{Basic Antiderivatives}
    \begin{enumerate}
      \item $\displaystyle\int e^{ax}\,dx = \frac{1}{a}e^{ax} + C$
      
      \item $\displaystyle\int \frac{1}{x}\,dx = \ln|x| + C$
      
      \item $\displaystyle\int \frac{1}{a^2 + x^2}\,dx = \frac{1}{a}\arctan\left(\frac{x}{a}\right) + C$
      
      \item $\displaystyle\int \frac{1}{\sqrt{a^2 - x^2}}\,dx = \arcsin\left(\frac{x}{a}\right) + C$
      
      \item $\displaystyle\int x^n e^{ax}\,dx$ requires integration by parts repeatedly
    \end{enumerate}
  \end{block}
\end{frame}

% Final slide
\begin{frame}
  \begin{center}
    {\Huge Thank You!}
    
    \vspace{1cm}
    
    {\Large Yolymatics Tutorials}
    
    \vspace{0.5cm}
    
    {\large www.yolymaticstutorials.com}
    
    \vspace{1cm}
    
    \textcolor{yolyaccent}{\Large Keep practicing!}
  \end{center}
\end{frame}

\end{document}
