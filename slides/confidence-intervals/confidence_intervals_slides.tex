\documentclass[aspectratio=169,11pt]{beamer}
\usetheme{Madrid}
\usecolortheme{default}

% Packages
\usepackage{amsmath}
\usepackage{amssymb}
\usepackage{tikz}
\usepackage{array}
\usepackage{multirow}
\usepackage{booktabs}
\usepackage{graphicx}

% Custom colors - Professional palette
\definecolor{yolyblue}{RGB}{13,71,161}
\definecolor{yolylightblue}{RGB}{25,118,210}
\definecolor{yolydarkblue}{RGB}{1,87,155}
\definecolor{yolyaccent}{RGB}{245,124,0}

\setbeamercolor{structure}{fg=yolyblue}
\setbeamercolor{title}{fg=white,bg=yolyblue}
\setbeamercolor{frametitle}{fg=white,bg=yolyblue}
\setbeamercolor{block title}{bg=yolylightblue,fg=white}
\setbeamercolor{block body}{bg=yolylightblue!10,fg=black}

% Professional footer
\setbeamertemplate{footline}
{
  \leavevmode%
  \hbox{%
  \begin{beamercolorbox}[wd=.333333\paperwidth,ht=2.25ex,dp=1ex,center]{author in head/foot}%
    \usebeamerfont{author in head/foot}Yolymatics Tutorials
  \end{beamercolorbox}%
  \begin{beamercolorbox}[wd=.333333\paperwidth,ht=2.25ex,dp=1ex,center]{title in head/foot}%
    \usebeamerfont{title in head/foot}Confidence Intervals
  \end{beamercolorbox}%
  \begin{beamercolorbox}[wd=.333333\paperwidth,ht=2.25ex,dp=1ex,right]{date in head/foot}%
    \usebeamerfont{date in head/foot}\insertshortdate{}\hspace*{2em}
    \insertframenumber{} / \inserttotalframenumber\hspace*{2ex} 
  \end{beamercolorbox}}%
  \vskip0pt%
}

% Remove navigation symbols
\setbeamertemplate{navigation symbols}{}

% Title information
\title{Confidence Intervals}
\subtitle{Statistical Inference and Estimation}
\author{Yolymatics Tutorials}
\institute{www.yolymaticstutorials.com}
\date{\today}

\begin{document}

% Title slide
\begin{frame}
\titlepage
\begin{center}
\vspace{0.3cm}
\small \textit{Professional Mathematics Education}
\end{center}
\end{frame}

% Table of contents
\begin{frame}{Overview}
\tableofcontents
\end{frame}

% ============================================================================
\section{Introduction to Confidence Intervals}
% ============================================================================

\begin{frame}{What is a Confidence Interval?}

\begin{block}{Definition}
A confidence interval is a range of values, computed from sample data, that is likely to contain the true population parameter with a specified level of confidence.
\end{block}

\vspace{0.5cm}

\textbf{General Form:}
$$\text{Point Estimate} \pm \text{Margin of Error}$$

\vspace{0.5cm}

\textbf{Key Components:}
\begin{itemize}
\item \textbf{Point Estimate}: Sample statistic (e.g., $\bar{x}$, $\hat{p}$)
\item \textbf{Margin of Error}: Uncertainty measure based on standard error and confidence level
\item \textbf{Confidence Level}: Probability that interval contains true parameter (e.g., 90\%, 95\%, 99\%)
\end{itemize}

\end{frame}

\begin{frame}{Interpretation of Confidence Intervals}

\textbf{Correct Interpretation:}

If we construct many $(1-\alpha)100\%$ confidence intervals from repeated samples, approximately $(1-\alpha)100\%$ of them will contain the true population parameter.

\vspace{0.5cm}

\textbf{Common Confidence Levels:}
\begin{itemize}
\item 90\% confidence level: $\alpha = 0.10$, $z_{\alpha/2} = 1.645$
\item 95\% confidence level: $\alpha = 0.05$, $z_{\alpha/2} = 1.96$
\item 99\% confidence level: $\alpha = 0.01$, $z_{\alpha/2} = 2.576$
\end{itemize}

\vspace{0.5cm}

\textbf{Trade-offs:}
\begin{itemize}
\item Higher confidence $\Rightarrow$ Wider interval
\item Larger sample size $\Rightarrow$ Narrower interval
\end{itemize}

\end{frame}

% ============================================================================
\section{Confidence Interval for Population Mean ($\mu$)}
% ============================================================================

\begin{frame}{CI for Mean: Known Population Variance ($\sigma^2$)}

\textbf{When $\sigma$ is known (rare in practice):}

$$\bar{x} \pm z_{\alpha/2} \cdot \frac{\sigma}{\sqrt{n}}$$

\vspace{0.5cm}

\textbf{Where:}
\begin{itemize}
\item $\bar{x}$ = sample mean
\item $z_{\alpha/2}$ = critical value from standard normal distribution
\item $\sigma$ = population standard deviation (known)
\item $n$ = sample size
\end{itemize}

\vspace{0.5cm}

\textbf{Assumptions:}
\begin{itemize}
\item Random sample
\item Population is normally distributed OR $n \geq 30$ (CLT)
\item $\sigma$ is known
\end{itemize}

\end{frame}

\begin{frame}{CI for Mean: Unknown Population Variance ($\sigma^2$)}

\textbf{When $\sigma$ is unknown (most common):}

$$\bar{x} \pm t_{\alpha/2, n-1} \cdot \frac{s}{\sqrt{n}}$$

\vspace{0.5cm}

\textbf{Where:}
\begin{itemize}
\item $\bar{x}$ = sample mean
\item $t_{\alpha/2, n-1}$ = critical value from t-distribution with $n-1$ degrees of freedom
\item $s$ = sample standard deviation
\item $n$ = sample size
\end{itemize}

\vspace{0.5cm}

\textbf{Assumptions:}
\begin{itemize}
\item Random sample
\item Population is approximately normally distributed (especially for small $n$)
\item $\sigma$ is unknown
\end{itemize}

\end{frame}

\begin{frame}{t-Distribution Critical Values}

\textbf{Common critical values for t-distribution:}

\begin{center}
\small
\begin{tabular}{|c|c|c|c|}
\hline
\textbf{df} & \textbf{90\% CI} & \textbf{95\% CI} & \textbf{99\% CI} \\
 & $(t_{0.05})$ & $(t_{0.025})$ & $(t_{0.005})$ \\
\hline
5 & 2.015 & 2.571 & 4.032 \\
\hline
10 & 1.812 & 2.228 & 3.169 \\
\hline
15 & 1.753 & 2.131 & 2.947 \\
\hline
20 & 1.725 & 2.086 & 2.845 \\
\hline
25 & 1.708 & 2.060 & 2.787 \\
\hline
30 & 1.697 & 2.042 & 2.750 \\
\hline
$\infty$ (z) & 1.645 & 1.96 & 2.576 \\
\hline
\end{tabular}
\end{center}

\vspace{0.3cm}

Note: As df $\to \infty$, t-distribution $\to$ standard normal (z)

\end{frame}

% ============================================================================
\section{Mean CI: Practice Problems}
% ============================================================================

\begin{frame}{Practice Problem 1: Battery Life (Known $\sigma$)}

A manufacturer claims their batteries have a mean life of 500 hours. A random sample of 36 batteries has a mean life of 485 hours. The population standard deviation is known to be 40 hours.

\vspace{0.3cm}

Construct a 95\% confidence interval for the true mean battery life.

\vspace{6cm}

\end{frame}

\begin{frame}{Problem 1: Working Space}

\vspace{10cm}

\end{frame}

\begin{frame}{Practice Problem 2: Student Heights (Unknown $\sigma$)}

A random sample of 25 university students has a mean height of 170 cm with a standard deviation of 8 cm.

\vspace{0.3cm}

Construct a 90\% confidence interval for the mean height of all university students.

\vspace{7cm}

\end{frame}

\begin{frame}{Problem 2: Working Space}

\vspace{10cm}

\end{frame}

\begin{frame}{Practice Problem 3: Fuel Efficiency}

An automotive engineer tests 15 cars and finds the mean fuel efficiency is 32.5 mpg with a standard deviation of 3.2 mpg.

\vspace{0.3cm}

Calculate a 99\% confidence interval for the true mean fuel efficiency.

\vspace{7cm}

\end{frame}

\begin{frame}{Problem 3: Working Space}

\vspace{10cm}

\end{frame}

\begin{frame}{Practice Problem 4: Test Scores}

A teacher records test scores for 20 students. The sample mean is 78.4 and the sample standard deviation is 12.6.

\vspace{0.3cm}

Construct a 95\% confidence interval for the true mean test score.

\vspace{7cm}

\end{frame}

\begin{frame}{Problem 4: Working Space}

\vspace{10cm}

\end{frame}

\begin{frame}{Practice Problem 5: Manufacturing Process}

A quality control inspector measures 40 components. The mean diameter is 5.02 mm with a standard deviation of 0.15 mm.

\vspace{0.3cm}

Find a 90\% confidence interval for the true mean diameter.

\vspace{7cm}

\end{frame}

\begin{frame}{Problem 5: Working Space}

\vspace{10cm}

\end{frame}

% ============================================================================
\section{Confidence Interval for Population Proportion ($p$)}
% ============================================================================

\begin{frame}{CI for Population Proportion}

\textbf{For large samples:}

$$\hat{p} \pm z_{\alpha/2} \sqrt{\frac{\hat{p}(1-\hat{p})}{n}}$$

\vspace{0.5cm}

\textbf{Where:}
\begin{itemize}
\item $\hat{p} = \frac{x}{n}$ = sample proportion
\item $x$ = number of successes in sample
\item $n$ = sample size
\item $z_{\alpha/2}$ = critical value from standard normal distribution
\end{itemize}

\vspace{0.5cm}

\textbf{Conditions:}
\begin{itemize}
\item Random sample
\item $n\hat{p} \geq 10$ and $n(1-\hat{p}) \geq 10$ (large sample approximation)
\item Population size $\geq 10n$ (or use finite population correction)
\end{itemize}

\end{frame}

% ============================================================================
\section{Proportion CI: Practice Problems}
% ============================================================================

\begin{frame}{Practice Problem 6: Voter Preference}

In a survey of 500 registered voters, 285 indicated they support a particular candidate.

\vspace{0.3cm}

Construct a 95\% confidence interval for the true proportion of voters who support this candidate.

\vspace{7cm}

\end{frame}

\begin{frame}{Problem 6: Working Space}

\vspace{10cm}

\end{frame}

\begin{frame}{Practice Problem 7: Product Defect Rate}

A quality inspector examines 200 products and finds 12 are defective.

\vspace{0.3cm}

Calculate a 90\% confidence interval for the true defect rate.

\vspace{7cm}

\end{frame}

\begin{frame}{Problem 7: Working Space}

\vspace{10cm}

\end{frame}

\begin{frame}{Practice Problem 8: Medical Treatment Success}

In a clinical trial, 145 out of 180 patients showed improvement after treatment.

\vspace{0.3cm}

Construct a 99\% confidence interval for the true success rate of the treatment.

\vspace{7cm}

\end{frame}

\begin{frame}{Problem 8: Working Space}

\vspace{10cm}

\end{frame}

\begin{frame}{Practice Problem 9: Student Satisfaction}

A university surveys 350 students, and 273 report being satisfied with campus facilities.

\vspace{0.3cm}

Find a 95\% confidence interval for the proportion of all students satisfied with campus facilities.

\vspace{7cm}

\end{frame}

\begin{frame}{Problem 9: Working Space}

\vspace{10cm}

\end{frame}

\begin{frame}{Practice Problem 10: Market Share}

A market research firm surveys 600 consumers. 228 currently use Brand X.

\vspace{0.3cm}

Construct a 90\% confidence interval for Brand X's true market share.

\vspace{7cm}

\end{frame}

\begin{frame}{Problem 10: Working Space}

\vspace{10cm}

\end{frame}

% ============================================================================
\section{Sample Size Determination}
% ============================================================================

\begin{frame}{Determining Sample Size}

\textbf{For estimating a mean (known $\sigma$):}

$$n = \left(\frac{z_{\alpha/2} \cdot \sigma}{E}\right)^2$$

where $E$ = desired margin of error

\vspace{0.5cm}

\textbf{For estimating a proportion:}

$$n = \hat{p}(1-\hat{p})\left(\frac{z_{\alpha/2}}{E}\right)^2$$

If no prior estimate of $\hat{p}$, use $\hat{p} = 0.5$ (most conservative)

\vspace{0.5cm}

\textbf{Always round up to next integer!}

\end{frame}

% ============================================================================
\section{Sample Size: Practice Problems}
% ============================================================================

\begin{frame}{Practice Problem 11: Required Sample Size for Mean}

A researcher wants to estimate the mean weight of adults with a margin of error of 2 kg at 95\% confidence. Previous studies suggest $\sigma = 15$ kg.

\vspace{0.3cm}

How many adults should be sampled?

\vspace{7cm}

\end{frame}

\begin{frame}{Problem 11: Working Space}

\vspace{10cm}

\end{frame}

\begin{frame}{Practice Problem 12: Required Sample Size for Proportion}

A political pollster wants to estimate voter support with a margin of error of 3\% at 95\% confidence. No prior estimate is available.

\vspace{0.3cm}

What sample size is required?

\vspace{7cm}

\end{frame}

\begin{frame}{Problem 12: Working Space}

\vspace{10cm}

\end{frame}

% ============================================================================
\section{Confidence Interval for Difference of Means}
% ============================================================================

\begin{frame}{CI for Difference of Two Means (Independent Samples)}

\textbf{When both $\sigma_1$ and $\sigma_2$ are unknown:}

$$(\bar{x}_1 - \bar{x}_2) \pm t_{\alpha/2, df} \cdot \sqrt{\frac{s_1^2}{n_1} + \frac{s_2^2}{n_2}}$$

\vspace{0.3cm}

\textbf{Degrees of freedom (Welch approximation):}

$$df = \frac{\left(\frac{s_1^2}{n_1} + \frac{s_2^2}{n_2}\right)^2}{\frac{(s_1^2/n_1)^2}{n_1-1} + \frac{(s_2^2/n_2)^2}{n_2-1}}$$

\vspace{0.3cm}

\textbf{Assumptions:}
\begin{itemize}
\item Independent random samples
\item Both populations approximately normal (especially for small samples)
\end{itemize}

\end{frame}

% ============================================================================
\section{Difference of Means: Practice Problems}
% ============================================================================

\begin{frame}{Practice Problem 13: Comparing Two Teaching Methods}

Method A: $n_1 = 30$, $\bar{x}_1 = 82.5$, $s_1 = 8.2$

Method B: $n_2 = 25$, $\bar{x}_2 = 78.3$, $s_2 = 9.5$

\vspace{0.3cm}

Construct a 95\% confidence interval for the difference in mean scores between the two methods.

\vspace{6cm}

\end{frame}

\begin{frame}{Problem 13: Working Space}

\vspace{10cm}

\end{frame}

\begin{frame}{Practice Problem 14: Drug Efficacy Comparison}

Drug A: $n_1 = 40$, $\bar{x}_1 = 15.2$ days, $s_1 = 3.8$ days

Drug B: $n_2 = 35$, $\bar{x}_2 = 18.5$ days, $s_2 = 4.2$ days

\vspace{0.3cm}

Find a 90\% confidence interval for the difference in mean recovery time.

\vspace{6cm}

\end{frame}

\begin{frame}{Problem 14: Working Space}

\vspace{10cm}

\end{frame}

% ============================================================================
\section{Confidence Interval for Difference of Proportions}
% ============================================================================

\begin{frame}{CI for Difference of Two Proportions}

\textbf{For large samples:}

$$(\hat{p}_1 - \hat{p}_2) \pm z_{\alpha/2} \sqrt{\frac{\hat{p}_1(1-\hat{p}_1)}{n_1} + \frac{\hat{p}_2(1-\hat{p}_2)}{n_2}}$$

\vspace{0.5cm}

\textbf{Where:}
\begin{itemize}
\item $\hat{p}_1 = \frac{x_1}{n_1}$ and $\hat{p}_2 = \frac{x_2}{n_2}$ = sample proportions
\item $n_1, n_2$ = sample sizes
\end{itemize}

\vspace{0.5cm}

\textbf{Conditions:}
\begin{itemize}
\item Independent random samples
\item $n_1\hat{p}_1 \geq 10$, $n_1(1-\hat{p}_1) \geq 10$
\item $n_2\hat{p}_2 \geq 10$, $n_2(1-\hat{p}_2) \geq 10$
\end{itemize}

\end{frame}

% ============================================================================
\section{Difference of Proportions: Practice Problems}
% ============================================================================

\begin{frame}{Practice Problem 15: Gender Wage Gap}

Sample 1 (Men): $n_1 = 200$, 156 earn above median

Sample 2 (Women): $n_2 = 180$, 126 earn above median

\vspace{0.3cm}

Construct a 95\% confidence interval for the difference in proportions.

\vspace{6cm}

\end{frame}

\begin{frame}{Problem 15: Working Space}

\vspace{10cm}

\end{frame}

\begin{frame}{Practice Problem 16: Treatment Comparison}

Treatment A: 82 successes out of 120 patients

Treatment B: 65 successes out of 100 patients

\vspace{0.3cm}

Find a 99\% confidence interval for the difference in success rates.

\vspace{6cm}

\end{frame}

\begin{frame}{Problem 16: Working Space}

\vspace{10cm}

\end{frame}

% ============================================================================
\section{Summary}
% ============================================================================

\begin{frame}{Confidence Intervals: Summary Table}

\begin{center}
\small
\begin{tabular}{|p{3cm}|p{6cm}|}
\hline
\textbf{Parameter} & \textbf{Confidence Interval Formula} \\
\hline
Mean ($\sigma$ known) & $\bar{x} \pm z_{\alpha/2} \cdot \frac{\sigma}{\sqrt{n}}$ \\
\hline
Mean ($\sigma$ unknown) & $\bar{x} \pm t_{\alpha/2, n-1} \cdot \frac{s}{\sqrt{n}}$ \\
\hline
Proportion & $\hat{p} \pm z_{\alpha/2} \sqrt{\frac{\hat{p}(1-\hat{p})}{n}}$ \\
\hline
Difference of Means & $(\bar{x}_1 - \bar{x}_2) \pm t_{\alpha/2, df} \cdot \sqrt{\frac{s_1^2}{n_1} + \frac{s_2^2}{n_2}}$ \\
\hline
Difference of Proportions & $(\hat{p}_1 - \hat{p}_2) \pm z_{\alpha/2} \sqrt{\frac{\hat{p}_1(1-\hat{p}_1)}{n_1} + \frac{\hat{p}_2(1-\hat{p}_2)}{n_2}}$ \\
\hline
\end{tabular}
\end{center}

\end{frame}

\begin{frame}{Key Points to Remember}

\begin{itemize}
\item Use \textbf{z-distribution} when $\sigma$ is known or for proportions
\item Use \textbf{t-distribution} when $\sigma$ is unknown (most common for means)
\item Check \textbf{assumptions} before constructing intervals
\item \textbf{Wider intervals} = higher confidence level
\item \textbf{Narrower intervals} = larger sample size
\item Always verify \textbf{normality} or \textbf{large sample conditions}
\item Round sample sizes \textbf{up} to ensure desired precision
\end{itemize}

\end{frame}

\begin{frame}
\begin{center}
\Huge \textcolor{yolyblue}{Thank You}

\vspace{1cm}

\Large Questions?

\vspace{1.5cm}

\normalsize
\textbf{Yolymatics Tutorials}

\textcolor{yolylightblue}{www.yolymaticstutorials.com}

\vspace{1cm}

\textit{Professional Mathematics Education}
\end{center}
\end{frame}

\end{document}
