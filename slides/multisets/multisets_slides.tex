\documentclass[aspectratio=169]{beamer}
\usetheme{Madrid}
\usecolortheme{default}

% Packages
\usepackage{amsmath}
\usepackage{amssymb}
\usepackage{tikz}
\usepackage{array}
\usepackage{booktabs}

% Define Yolymatics colors
\definecolor{yolyblue}{RGB}{13,71,161}
\definecolor{yolylightblue}{RGB}{25,118,210}
\definecolor{yolyaccent}{RGB}{245,124,0}

% Set theme colors
\setbeamercolor{structure}{fg=yolyblue}
\setbeamercolor{palette primary}{bg=yolyblue,fg=white}
\setbeamercolor{palette secondary}{bg=yolylightblue,fg=white}
\setbeamercolor{palette tertiary}{bg=yolyaccent,fg=white}
\setbeamercolor{frametitle}{bg=yolyblue,fg=white}

% Footer
\setbeamertemplate{footline}{
  \leavevmode%
  \hbox{%
  \begin{beamercolorbox}[wd=.333333\paperwidth,ht=2.25ex,dp=1ex,center]{author in head/foot}%
    \usebeamerfont{author in head/foot}\insertshortauthor
  \end{beamercolorbox}%
  \begin{beamercolorbox}[wd=.333333\paperwidth,ht=2.25ex,dp=1ex,center]{title in head/foot}%
    \usebeamerfont{title in head/foot}\insertshorttitle
  \end{beamercolorbox}%
  \begin{beamercolorbox}[wd=.333333\paperwidth,ht=2.25ex,dp=1ex,right]{date in head/foot}%
    \usebeamerfont{date in head/foot}\insertshortdate{}\hspace*{2em}
    \insertframenumber{} / \inserttotalframenumber\hspace*{2ex} 
  \end{beamercolorbox}}%
  \vskip0pt%
}

% Title information
\title{Multisets}
\subtitle{Combinations with Repetition}
\author{Yolymatics Tutorials}
\institute{www.yolymaticstutorials.com}
\date{\today}

\begin{document}

% Title slide
\begin{frame}
  \titlepage
\end{frame}

% Table of contents
\begin{frame}{Outline}
  \tableofcontents
\end{frame}

% Section 1: Introduction
\section{Introduction to Multisets}

\begin{frame}{What is a Multiset?}
  \begin{block}{Definition}
    A \textbf{multiset} (or \textbf{bag}) is a collection of objects where repetition of elements is allowed, but order does not matter.
  \end{block}
  
  \vspace{0.5cm}
  
  \begin{block}{Comparison with Sets}
    \begin{itemize}
      \item \textbf{Set}: $\{a, b, c\}$ - each element appears at most once
      \item \textbf{Multiset}: $\{a, a, b, c, c, c\}$ - elements can repeat
      \item Notation: Sometimes written as $\{a^2, b, c^3\}$ to show multiplicities
    \end{itemize}
  \end{block}
  
  \begin{alertblock}{Key Property}
    The order of elements does NOT matter: $\{a, a, b\} = \{a, b, a\} = \{b, a, a\}$
  \end{alertblock}
\end{frame}

\begin{frame}{Multisets vs Combinations vs Permutations}
  \begin{block}{Key Distinctions}
    \begin{tabular}{llll}
      \toprule
      \textbf{Concept} & \textbf{Order?} & \textbf{Repetition?} & \textbf{Example} \\
      \midrule
      Permutations & Matters & No & $ABC \neq BAC$ \\
      Combinations & Doesn't & No & $\{A,B,C\} = \{C,A,B\}$ \\
      \textbf{Multisets} & \textbf{Doesn't} & \textbf{Yes} & $\{A,A,B\} = \{B,A,A\}$ \\
      Permutations w/ Rep & Matters & Yes & $AAB \neq ABA$ \\
      \bottomrule
    \end{tabular}
  \end{block}
  
  \vspace{0.5cm}
  
  \begin{exampleblock}{Real-World Examples}
    \begin{itemize}
      \item Selecting fruits: 2 apples, 1 banana, 3 oranges
      \item Distributing identical items to distinct boxes
      \item Choosing toppings for pizza (can repeat)
    \end{itemize}
  \end{exampleblock}
\end{frame}

\begin{frame}{The Fundamental Multiset Formula}
  \begin{block}{Problem Statement}
    How many ways can we select $r$ objects from $n$ types of objects, where repetition is allowed and order does not matter?
  \end{block}
  
  \vspace{0.5cm}
  
  \begin{block}{Formula: Combinations with Repetition}
    $$\left(\!\!\binom{n}{r}\!\!\right) = \binom{n+r-1}{r} = \binom{n+r-1}{n-1} = \frac{(n+r-1)!}{r!(n-1)!}$$
    
    where:
    \begin{itemize}
      \item $n$ = number of types (distinct objects to choose from)
      \item $r$ = number of selections (objects to be selected)
    \end{itemize}
  \end{block}
  
  \begin{alertblock}{Notation}
    $\left(\!\!\binom{n}{r}\!\!\right)$ is read as "n multichoose r" or "multiset coefficient"
  \end{alertblock}
\end{frame}

\begin{frame}{Stars and Bars Method}
  \begin{block}{Visual Representation}
    The multiset problem is equivalent to placing $r$ identical stars into $n$ distinct bins using $n-1$ bars as separators.
  \end{block}
  
  \begin{exampleblock}{Example: Choose 5 items from 3 types}
    Select 5 objects from types $\{A, B, C\}$
    
    \vspace{0.3cm}
    
    Representation: $\star \star | \star \star \star |$ means 2 of type A, 3 of type B, 0 of type C
    
    \vspace{0.3cm}
    
    Total arrangements: We have 5 stars and 2 bars to arrange
    
    $$\binom{5+2}{2} = \binom{7}{2} = 21 \text{ ways}$$
  \end{exampleblock}
  
  \begin{alertblock}{Key Insight}
    We arrange $r$ stars and $(n-1)$ bars, giving $\binom{n+r-1}{r}$ or $\binom{n+r-1}{n-1}$
  \end{alertblock}
\end{frame}

% Section 2: Basic Multiset Problems
\section{Basic Multiset Problems}

\begin{frame}{Problem 1: Simple Multiset Selection}
  \begin{block}{Problem}
    A bakery offers 4 types of cookies: chocolate chip, oatmeal, sugar, and peanut butter. You want to buy 6 cookies. How many different selections can you make?
    
    \textit{Note: You can choose multiple cookies of the same type.}
  \end{block}
\end{frame}

\begin{frame}{Working Space}
  \vspace{6cm}
\end{frame}

\begin{frame}{Problem 2: Distributing Identical Objects}
  \begin{block}{Problem}
    In how many ways can 10 identical candies be distributed among 3 children?
    
    \textit{Note: A child can receive 0, 1, 2, ..., or all 10 candies.}
  \end{block}
\end{frame}

\begin{frame}{Working Space}
  \vspace{6cm}
\end{frame}

\begin{frame}{Problem 3: Non-negative Integer Solutions}
  \begin{block}{Problem}
    Find the number of non-negative integer solutions to the equation:
    $$x_1 + x_2 + x_3 + x_4 = 12$$
  \end{block}
\end{frame}

\begin{frame}{Working Space}
  \vspace{6cm}
\end{frame}

\begin{frame}{Problem 4: Coin Selection}
  \begin{block}{Problem}
    A vending machine accepts pennies, nickels, dimes, and quarters. How many different ways can you put 8 coins into the machine?
    
    \textit{Note: The coins need not have the same value.}
  \end{block}
\end{frame}

\begin{frame}{Working Space}
  \vspace{6cm}
\end{frame}

\begin{frame}{Problem 5: Letter Repetition}
  \begin{block}{Problem}
    How many 5-letter "words" can be formed using only the letters A, B, and C, where order does not matter?
    
    \textit{Example: AAABC is the same as AABCA and BAAAC.}
  \end{block}
\end{frame}

\begin{frame}{Working Space}
  \vspace{6cm}
\end{frame}

% Section 3: Multisets with Constraints
\section{Multisets with Constraints}

\begin{frame}{Problem 6: Positive Integer Solutions}
  \begin{block}{Problem}
    Find the number of \textbf{positive} integer solutions to:
    $$x_1 + x_2 + x_3 = 10$$
    
    \textit{Hint: Each variable must be at least 1.}
  \end{block}
\end{frame}

\begin{frame}{Working Space}
  \vspace{6cm}
\end{frame}

\begin{frame}{Problem 7: Minimum Constraints}
  \begin{block}{Problem}
    How many ways can 15 identical books be distributed among 4 students if each student must receive at least 2 books?
  \end{block}
\end{frame}

\begin{frame}{Working Space}
  \vspace{6cm}
\end{frame}

\begin{frame}{Problem 8: Upper Bound Constraints}
  \begin{block}{Problem}
    Find the number of non-negative integer solutions to:
    $$x_1 + x_2 + x_3 = 8$$
    where $x_1 \leq 3$, $x_2 \leq 4$, and $x_3 \leq 5$.
    
    \textit{Hint: Use inclusion-exclusion principle.}
  \end{block}
\end{frame}

\begin{frame}{Working Space}
  \vspace{6cm}
\end{frame}

\begin{frame}{Problem 9: Mixed Constraints}
  \begin{block}{Problem}
    Find the number of integer solutions to:
    $$x_1 + x_2 + x_3 + x_4 = 20$$
    where $x_1 \geq 2$, $x_2 \geq 0$, $x_3 \geq 1$, and $x_4 \geq 3$.
  \end{block}
\end{frame}

\begin{frame}{Working Space}
  \vspace{6cm}
\end{frame}

\begin{frame}{Problem 10: Even Distribution Constraint}
  \begin{block}{Problem}
    How many ways can 12 identical items be distributed among 3 boxes such that each box contains an even number of items (possibly 0)?
    
    \textit{Hint: Let $x_i = 2y_i$ where $y_i \geq 0$.}
  \end{block}
\end{frame}

\begin{frame}{Working Space}
  \vspace{6cm}
\end{frame}

% Section 4: Advanced Applications
\section{Advanced Multiset Problems}

\begin{frame}{Problem 11: Generating Functions Application}
  \begin{block}{Problem}
    In how many ways can you make change for \$1.00 using only pennies, nickels, dimes, and quarters?
    
    \textit{Note: Find non-negative integer solutions to $p + 5n + 10d + 25q = 100$.}
  \end{block}
\end{frame}

\begin{frame}{Working Space}
  \vspace{6cm}
\end{frame}

\begin{frame}{Problem 12: Partition with Distinct Parts}
  \begin{block}{Problem}
    How many ways can you select a multiset of size 6 from the set $\{1, 2, 3, 4, 5\}$?
    
    \textit{Note: Each number can be selected multiple times.}
  \end{block}
\end{frame}

\begin{frame}{Working Space}
  \vspace{6cm}
\end{frame}

\begin{frame}{Problem 13: Polynomial Coefficient}
  \begin{block}{Problem}
    Find the coefficient of $x^{10}$ in the expansion of:
    $$(1 + x + x^2 + x^3 + \cdots)^4$$
    
    \textit{Hint: This is equivalent to a multiset problem.}
  \end{block}
\end{frame}

\begin{frame}{Working Space}
  \vspace{6cm}
\end{frame}

\begin{frame}{Problem 14: Dice Combinations}
  \begin{block}{Problem}
    How many different unordered outcomes are possible when rolling 5 distinguishable dice?
    
    \textit{Example: Rolling (1,1,3,4,6) is the same as (1,3,1,6,4) in an unordered collection.}
  \end{block}
\end{frame}

\begin{frame}{Working Space}
  \vspace{6cm}
\end{frame}

\begin{frame}{Problem 15: Bounded Selections}
  \begin{block}{Problem}
    A fruit stand has apples, bananas, and cherries. You want to buy exactly 8 pieces of fruit. However, the stand only has 3 bananas available. How many different selections are possible?
  \end{block}
\end{frame}

\begin{frame}{Working Space}
  \vspace{6cm}
\end{frame}

% Section 5: Comparison and Counting Problems
\section{Counting Method Identification}

\begin{frame}{Problem 16: Identify the Method}
  \begin{block}{Problem}
    For each scenario, determine whether it is a multiset problem, regular combination, or permutation:
    
    \begin{enumerate}
      \item Selecting 4 cards from a standard deck (52 cards)
      \item Arranging 5 books on a shelf
      \item Choosing 10 donuts from 6 varieties
      \item Forming a 3-digit PIN code
    \end{enumerate}
  \end{block}
\end{frame}

\begin{frame}{Working Space}
  \vspace{6cm}
\end{frame}

\begin{frame}{Problem 17: Ice Cream Scoops}
  \begin{block}{Problem}
    An ice cream shop has 12 flavors. You order a bowl with 4 scoops. How many different bowls are possible if:
    
    \begin{enumerate}
      \item Order matters and repetition is not allowed
      \item Order doesn't matter and repetition is not allowed
      \item Order doesn't matter and repetition is allowed
    \end{enumerate}
  \end{block}
\end{frame}

\begin{frame}{Working Space}
  \vspace{6cm}
\end{frame}

\begin{frame}{Problem 18: Committee with Repeated Members}
  \begin{block}{Problem}
    A committee of 5 people is to be formed from 8 candidates, where a person can be selected multiple times (representing multiple votes or roles). How many different committees can be formed?
  \end{block}
\end{frame}

\begin{frame}{Working Space}
  \vspace{6cm}
\end{frame}

% Section 6: Multiset Theory and Properties
\section{Theory and Formulas}

\begin{frame}{Formula 1: Basic Multiset Coefficient}
  \begin{block}{Statement}
    The number of multisets of size $r$ chosen from $n$ types is:
    $$\left(\!\!\binom{n}{r}\!\!\right) = \binom{n+r-1}{r} = \binom{n+r-1}{n-1}$$
  \end{block}
  
  \begin{exampleblock}{Alternative Forms}
    \begin{itemize}
      \item $\displaystyle\frac{(n+r-1)!}{r!(n-1)!}$ - factorial form
      
      \item $\displaystyle\frac{(n+r-1)(n+r-2)\cdots(n)}{r!}$ - product form
      
      \item Number of non-negative integer solutions to $x_1 + x_2 + \cdots + x_n = r$
    \end{itemize}
  \end{exampleblock}
\end{frame}

\begin{frame}{Formula 2: Positive Integer Solutions}
  \begin{block}{Statement}
    The number of \textbf{positive} integer solutions to:
    $$x_1 + x_2 + \cdots + x_n = r$$
    where each $x_i \geq 1$, is:
    $$\binom{r-1}{n-1} = \binom{r-1}{r-n}$$
  \end{block}
  
  \begin{alertblock}{Derivation}
    Substitute $y_i = x_i - 1$ where $y_i \geq 0$:
    $$y_1 + y_2 + \cdots + y_n = r - n$$
    
    This gives $\binom{(r-n)+n-1}{n-1} = \binom{r-1}{n-1}$ non-negative solutions.
  \end{alertblock}
\end{frame}

\begin{frame}{Formula 3: General Minimum Constraints}
  \begin{block}{Statement}
    The number of integer solutions to $x_1 + x_2 + \cdots + x_n = r$ where $x_i \geq m_i$ for each $i$:
    
    $$\binom{r - (m_1 + m_2 + \cdots + m_n) + n - 1}{n-1}$$
  \end{block}
  
  \begin{block}{Method}
    \begin{enumerate}
      \item Substitute $y_i = x_i - m_i$ so that $y_i \geq 0$
      \item The equation becomes: $y_1 + y_2 + \cdots + y_n = r - \sum m_i$
      \item Apply the basic multiset formula to the $y_i$ variables
    \end{enumerate}
  \end{block}
\end{frame}

\begin{frame}{Formula 4: Inclusion-Exclusion for Upper Bounds}
  \begin{block}{Statement}
    For solutions to $x_1 + x_2 + \cdots + x_n = r$ with upper bounds $x_i \leq u_i$:
    
    Use the Inclusion-Exclusion Principle:
    $$\sum_{\text{all}} - \sum_{\text{violate 1}} + \sum_{\text{violate 2}} - \cdots$$
  \end{block}
  
  \begin{exampleblock}{Example: $x_1 + x_2 = 10$ with $x_1 \leq 6$, $x_2 \leq 7$}
    \begin{align*}
    \text{Total} &= \binom{11}{1} = 11 \\
    \text{Subtract } x_1 \geq 7 &= \binom{4}{1} = 4 \\
    \text{Subtract } x_2 \geq 8 &= \binom{3}{1} = 3 \\
    \text{Add back } x_1 \geq 7, x_2 \geq 8 &= 0 \\
    \text{Answer} &= 11 - 4 - 3 + 0 = 4
    \end{align*}
  \end{exampleblock}
\end{frame}

\begin{frame}{Formula 5: Stars and Bars Visualization}
  \begin{block}{Interpretation}
    To distribute $r$ identical objects into $n$ distinct bins:
    \begin{itemize}
      \item Represent objects as $r$ stars: $\star \star \star \cdots \star$
      \item Use $n-1$ bars to create $n$ compartments: $|$
      \item Arrange $r$ stars and $n-1$ bars in a line
    \end{itemize}
    
    Total arrangements: $\binom{r + (n-1)}{r} = \binom{r+n-1}{n-1}$
  \end{block}
  
  \begin{exampleblock}{Example}
    Distribute 5 items into 3 bins: $\star \star | \star | \star \star$
    
    This represents: Bin 1 gets 2, Bin 2 gets 1, Bin 3 gets 2
    
    Total ways: $\binom{5+3-1}{3-1} = \binom{7}{2} = 21$
  \end{exampleblock}
\end{frame}

\begin{frame}{Formula 6: Generating Functions}
  \begin{block}{Statement}
    The generating function for selecting $r$ objects from $n$ types with repetition is:
    $$(1 + x + x^2 + x^3 + \cdots)^n = \frac{1}{(1-x)^n}$$
    
    The coefficient of $x^r$ gives the number of multisets of size $r$.
  \end{block}
  
  \begin{block}{Power Series Expansion}
    $$\frac{1}{(1-x)^n} = \sum_{r=0}^{\infty} \binom{n+r-1}{r} x^r$$
    
    The coefficient of $x^r$ is $\binom{n+r-1}{r}$, confirming the multiset formula.
  \end{block}
\end{frame}

\begin{frame}{Property 1: Multiset Symmetry}
  \begin{block}{Statement}
    $$\left(\!\!\binom{n}{r}\!\!\right) = \binom{n+r-1}{r} = \binom{n+r-1}{n-1}$$
    
    This symmetry shows that selecting $r$ items from $n$ types is equivalent to distributing $r$ items into $n$ bins.
  \end{block}
  
  \begin{alertblock}{Interpretation}
    \begin{itemize}
      \item $\binom{n+r-1}{r}$ - choose positions for $r$ stars among $r$ stars and $n-1$ bars
      \item $\binom{n+r-1}{n-1}$ - choose positions for $n-1$ bars among $r$ stars and $n-1$ bars
    \end{itemize}
  \end{alertblock}
\end{frame}

\begin{frame}{Property 2: Relationship to Regular Combinations}
  \begin{block}{Comparison}
    \begin{itemize}
      \item \textbf{Regular combinations}: $\binom{n}{r}$ - choose $r$ from $n$ distinct objects, no repetition
      
      \item \textbf{Multisets}: $\left(\!\!\binom{n}{r}\!\!\right) = \binom{n+r-1}{r}$ - choose $r$ from $n$ types, repetition allowed
    \end{itemize}
  \end{block}
  
  \begin{exampleblock}{Example Comparison}
    \begin{itemize}
      \item $\binom{4}{3} = 4$ - choose 3 from 4 distinct items: $\{A,B,C\}, \{A,B,D\}, \{A,C,D\}, \{B,C,D\}$
      
      \item $\binom{4+3-1}{3} = \binom{6}{3} = 20$ - choose 3 from 4 types with repetition allowed
    \end{itemize}
  \end{exampleblock}
\end{frame}

\begin{frame}{Property 3: Special Cases}
  \begin{block}{Important Special Values}
    \begin{enumerate}
      \item $\left(\!\!\binom{n}{0}\!\!\right) = \binom{n-1}{0} = 1$ - one way to select nothing
      
      \item $\left(\!\!\binom{n}{1}\!\!\right) = \binom{n}{1} = n$ - $n$ ways to select one item
      
      \item $\left(\!\!\binom{1}{r}\!\!\right) = \binom{r}{r} = 1$ - one way to select $r$ items from 1 type
      
      \item $\left(\!\!\binom{2}{r}\!\!\right) = \binom{r+1}{r} = r+1$ - partitions of $r$ into at most 2 parts
    \end{enumerate}
  \end{block}
\end{frame}

\begin{frame}{Common Problem Types Summary}
  \begin{block}{Problem Type Recognition}
    \begin{tabular}{ll}
      \toprule
      \textbf{Problem Type} & \textbf{Formula} \\
      \midrule
      Distribute $r$ identical to $n$ distinct & $\binom{n+r-1}{r}$ \\
      Non-negative solutions: $\sum x_i = r$ & $\binom{n+r-1}{n-1}$ \\
      Positive solutions: $\sum x_i = r$ & $\binom{r-1}{n-1}$ \\
      With minimums: $x_i \geq m_i$ & $\binom{r-\sum m_i + n-1}{n-1}$ \\
      With maximums: $x_i \leq u_i$ & Inclusion-Exclusion \\
      Coefficient in $(1+x+x^2+\cdots)^n$ & $\binom{n+r-1}{r}$ \\
      \bottomrule
    \end{tabular}
  \end{block}
\end{frame}

% Section 7: Reference Formulas
\section{Quick Reference}

\begin{frame}{Quick Formula Reference}
  \begin{block}{Essential Formulas}
    \begin{enumerate}
      \item \textbf{Basic Multiset}: $\left(\!\!\binom{n}{r}\!\!\right) = \binom{n+r-1}{r} = \frac{(n+r-1)!}{r!(n-1)!}$
      
      \item \textbf{Positive Integer Solutions}: $x_1 + \cdots + x_n = r$ with $x_i \geq 1$: $\binom{r-1}{n-1}$
      
      \item \textbf{With Minimums}: $x_i \geq m_i$: Let $S = \sum m_i$, answer is $\binom{r-S+n-1}{n-1}$
      
      \item \textbf{Stars and Bars}: $r$ stars, $n-1$ bars: $\binom{r+n-1}{n-1}$
      
      \item \textbf{Regular Combination}: $\binom{n}{r} = \frac{n!}{r!(n-r)!}$ (no repetition)
      
      \item \textbf{Permutation}: $P(n,r) = \frac{n!}{(n-r)!}$ (order matters, no repetition)
    \end{enumerate}
  \end{block}
\end{frame}

% Final slide
\begin{frame}
  \begin{center}
    {\Huge Thank You!}
    
    \vspace{1cm}
    
    {\Large Yolymatics Tutorials}
    
    \vspace{0.5cm}
    
    {\large www.yolymaticstutorials.com}
    
    \vspace{1cm}
    
    \textcolor{yolyaccent}{\Large Keep counting!}
  \end{center}
\end{frame}

\end{document}
